%%% Шаблон %%%
\DeclareRobustCommand{\fixme}{\textcolor{red}}  % решаем проблему превращения
                                % названия цвета в результате \MakeUppercase,
                                % http://tex.stackexchange.com/a/187930,
                                % \DeclareRobustCommand protects \fixme
                                % from expanding inside \MakeUppercase
\AtBeginDocument{%
    \setlength{\parindent}{2.5em}                   % Абзацный отступ. Должен быть одинаковым по всему тексту и равен пяти знакам (ГОСТ Р 7.0.11-2011, 5.3.7).
}

%%% Таблицы %%%
\DeclareCaptionLabelSeparator{tabsep}{\tablabelsep} % нумерация таблиц
\DeclareCaptionFormat{split}{\splitformatlabel#1\par\splitformattext#3}

\captionsetup[table]{
        format=\tabformat,                % формат подписи (plain|hang)
        font=normal,                      % нормальные размер, цвет, стиль шрифта
        skip=.0pt,                        % отбивка под подписью
        parskip=.0pt,                     % отбивка между параграфами подписи
        position=above,                   % положение подписи
        justification=\tabjust,           % центровка
        indent=\tabindent,                % смещение строк после первой
        labelsep=tabsep,                  % разделитель
        singlelinecheck=\tabsinglecenter, % не выравнивать по центру, если умещается в одну строку
}

%%% Рисунки %%%
\DeclareCaptionLabelSeparator{figsep}{\figlabelsep} % нумерация рисунков

\captionsetup[figure]{
        format=plain,                     % формат подписи (plain|hang)
        font=normal,                      % нормальные размер, цвет, стиль шрифта
        skip=.0pt,                        % отбивка под подписью
        parskip=.0pt,                     % отбивка между параграфами подписи
        position=below,                   % положение подписи
        singlelinecheck=true,             % выравнивание по центру, если умещается в одну строку
        justification=centerlast,         % центровка
        labelsep=figsep,                  % разделитель
}

%%% Подписи подрисунков %%%
\DeclareCaptionSubType{figure}
\renewcommand\thesubfigure{\asbuk{subfigure}} % нумерация подрисунков
\DeclareCaptionFont{norm}{\fontsize{12pt}{16pt}\selectfont}
\newcommand{\subfigureskip}{0.pt}

\captionsetup[subfloat]{
        labelfont=norm,                 % нормальный размер подписей подрисунков
        textfont=norm,                  % нормальный размер подписей подрисунков
        labelsep=space,                 % разделитель
        labelformat=brace,              % одна скобка справа от номера
        justification=centering,        % центровка
        singlelinecheck=true,           % выравнивание по центру, если умещается в одну строку
        skip=\subfigureskip,            % отбивка над подписью
        parskip=.0pt,                   % отбивка между параграфами подписи
        position=below,                 % положение подписи
}

%%% Настройки ссылок на рисунки, таблицы и др. %%%
% команды \cref...format отвечают за форматирование при помощи команды \cref
% команды \labelcref...format отвечают за форматирование при помощи команды \labelcref

\ifpresentation
\else
    \crefdefaultlabelformat{#2#1#3}

    % Уравнение
    \crefformat{equation}{(#2#1#3)} % одиночная ссылка с приставкой
    \labelcrefformat{equation}{(#2#1#3)} % одиночная ссылка без приставки
    \crefrangeformat{equation}{(#3#1#4) \cyrdash~(#5#2#6)} % диапазон ссылок с приставкой
    \labelcrefrangeformat{equation}{(#3#1#4) \cyrdash~(#5#2#6)} % диапазон ссылок без приставки
    \crefmultiformat{equation}{(#2#1#3)}{ и~(#2#1#3)}{, (#2#1#3)}{ и~(#2#1#3)} % перечисление ссылок с приставкой
    \labelcrefmultiformat{equation}{(#2#1#3)}{ и~(#2#1#3)}{, (#2#1#3)}{ и~(#2#1#3)} % перечисление без приставки

    % Подуравнение
    \crefformat{subequation}{(#2#1#3)} % одиночная ссылка с приставкой
    \labelcrefformat{subequation}{(#2#1#3)} % одиночная ссылка без приставки
    \crefrangeformat{subequation}{(#3#1#4) \cyrdash~(#5#2#6)} % диапазон ссылок с приставкой
    \labelcrefrangeformat{subequation}{(#3#1#4) \cyrdash~(#5#2#6)} % диапазон ссылок без приставки
    \crefmultiformat{subequation}{(#2#1#3)}{ и~(#2#1#3)}{, (#2#1#3)}{ и~(#2#1#3)} % перечисление ссылок с приставкой
    \labelcrefmultiformat{subequation}{(#2#1#3)}{ и~(#2#1#3)}{, (#2#1#3)}{ и~(#2#1#3)} % перечисление без приставки

    % Глава
    \crefformat{chapter}{#2#1#3} % одиночная ссылка с приставкой
    \labelcrefformat{chapter}{#2#1#3} % одиночная ссылка без приставки
    \crefrangeformat{chapter}{#3#1#4 \cyrdash~#5#2#6} % диапазон ссылок с приставкой
    \labelcrefrangeformat{chapter}{#3#1#4 \cyrdash~#5#2#6} % диапазон ссылок без приставки
    \crefmultiformat{chapter}{#2#1#3}{ и~#2#1#3}{, #2#1#3}{ и~#2#1#3} % перечисление ссылок с приставкой
    \labelcrefmultiformat{chapter}{#2#1#3}{ и~#2#1#3}{, #2#1#3}{ и~#2#1#3} % перечисление без приставки

    % Параграф
    \crefformat{section}{#2#1#3} % одиночная ссылка с приставкой
    \labelcrefformat{section}{#2#1#3} % одиночная ссылка без приставки
    \crefrangeformat{section}{#3#1#4 \cyrdash~#5#2#6} % диапазон ссылок с приставкой
    \labelcrefrangeformat{section}{#3#1#4 \cyrdash~#5#2#6} % диапазон ссылок без приставки
    \crefmultiformat{section}{#2#1#3}{ и~#2#1#3}{, #2#1#3}{ и~#2#1#3} % перечисление ссылок с приставкой
    \labelcrefmultiformat{section}{#2#1#3}{ и~#2#1#3}{, #2#1#3}{ и~#2#1#3} % перечисление без приставки

    % Приложение
    \crefformat{appendix}{#2#1#3} % одиночная ссылка с приставкой
    \labelcrefformat{appendix}{#2#1#3} % одиночная ссылка без приставки
    \crefrangeformat{appendix}{#3#1#4 \cyrdash~#5#2#6} % диапазон ссылок с приставкой
    \labelcrefrangeformat{appendix}{#3#1#4 \cyrdash~#5#2#6} % диапазон ссылок без приставки
    \crefmultiformat{appendix}{#2#1#3}{ и~#2#1#3}{, #2#1#3}{ и~#2#1#3} % перечисление ссылок с приставкой
    \labelcrefmultiformat{appendix}{#2#1#3}{ и~#2#1#3}{, #2#1#3}{ и~#2#1#3} % перечисление без приставки

    % Рисунок
    \crefformat{figure}{#2#1#3} % одиночная ссылка с приставкой
    \labelcrefformat{figure}{#2#1#3} % одиночная ссылка без приставки
    \crefrangeformat{figure}{#3#1#4 \cyrdash~#5#2#6} % диапазон ссылок с приставкой
    \labelcrefrangeformat{figure}{#3#1#4 \cyrdash~#5#2#6} % диапазон ссылок без приставки
    \crefmultiformat{figure}{#2#1#3}{ и~#2#1#3}{, #2#1#3}{ и~#2#1#3} % перечисление ссылок с приставкой
    \labelcrefmultiformat{figure}{#2#1#3}{ и~#2#1#3}{, #2#1#3}{ и~#2#1#3} % перечисление без приставки

    % Таблица
    \crefformat{table}{#2#1#3} % одиночная ссылка с приставкой
    \labelcrefformat{table}{#2#1#3} % одиночная ссылка без приставки
    \crefrangeformat{table}{#3#1#4 \cyrdash~#5#2#6} % диапазон ссылок с приставкой
    \labelcrefrangeformat{table}{#3#1#4 \cyrdash~#5#2#6} % диапазон ссылок без приставки
    \crefmultiformat{table}{#2#1#3}{ и~#2#1#3}{, #2#1#3}{ и~#2#1#3} % перечисление ссылок с приставкой
    \labelcrefmultiformat{table}{#2#1#3}{ и~#2#1#3}{, #2#1#3}{ и~#2#1#3} % перечисление без приставки

    % Листинг
    \crefformat{lstlisting}{#2#1#3} % одиночная ссылка с приставкой
    \labelcrefformat{lstlisting}{#2#1#3} % одиночная ссылка без приставки
    \crefrangeformat{lstlisting}{#3#1#4 \cyrdash~#5#2#6} % диапазон ссылок с приставкой
    \labelcrefrangeformat{lstlisting}{#3#1#4 \cyrdash~#5#2#6} % диапазон ссылок без приставки
    \crefmultiformat{lstlisting}{#2#1#3}{ и~#2#1#3}{, #2#1#3}{ и~#2#1#3} % перечисление ссылок с приставкой
    \labelcrefmultiformat{lstlisting}{#2#1#3}{ и~#2#1#3}{, #2#1#3}{ и~#2#1#3} % перечисление без приставки

    % Листинг
    \crefformat{ListingEnv}{#2#1#3} % одиночная ссылка с приставкой
    \labelcrefformat{ListingEnv}{#2#1#3} % одиночная ссылка без приставки
    \crefrangeformat{ListingEnv}{#3#1#4 \cyrdash~#5#2#6} % диапазон ссылок с приставкой
    \labelcrefrangeformat{ListingEnv}{#3#1#4 \cyrdash~#5#2#6} % диапазон ссылок без приставки
    \crefmultiformat{ListingEnv}{#2#1#3}{ и~#2#1#3}{, #2#1#3}{ и~#2#1#3} % перечисление ссылок с приставкой
    \labelcrefmultiformat{ListingEnv}{#2#1#3}{ и~#2#1#3}{, #2#1#3}{ и~#2#1#3} % перечисление без приставки
\fi

%%% Настройки гиперссылок %%%
\ifluatex
    \hypersetup{
        unicode,                % Unicode encoded PDF strings
    }
\fi

\hypersetup{
    linktocpage=true,           % ссылки с номера страницы в оглавлении, списке таблиц и списке рисунков
%    linktoc=all,                % both the section and page part are links
%    pdfpagelabels=false,        % set PDF page labels (true|false)
    plainpages=false,           % Forces page anchors to be named by the Arabic form  of the page number, rather than the formatted form
    colorlinks,                 % ссылки отображаются раскрашенным текстом, а не раскрашенным прямоугольником, вокруг текста
    linkcolor={linkcolor},      % цвет ссылок типа ref, eqref и подобных
    citecolor={citecolor},      % цвет ссылок-цитат
    urlcolor={urlcolor},        % цвет гиперссылок
%    hidelinks,                  % Hide links (removing color and border)
    pdftitle={PhD Thesis},    % Заголовок
    pdfauthor={Author},  % Автор
    pdfsubject={01.04.05 Optics},      % Тема
%    pdfcreator={Создатель},     % Создатель, Приложение
%    pdfproducer={Производитель},% Производитель, Производитель PDF
    pdfkeywords={physics},    % Ключевые слова
    pdflang={ru},
}
\ifnumequal{\value{draft}}{1}{% Черновик
    \hypersetup{
        draft,
    }
}{}

%%% Списки %%%
% Используем короткое тире (endash) для ненумерованных списков (ГОСТ 2.105-95, пункт 4.1.7, требует дефиса, но так лучше смотрится)
\renewcommand{\labelitemi}{\normalfont\bfseries{--}}

% Перечисление строчными буквами латинского алфавита (ГОСТ 2.105-95, 4.1.7)
%\renewcommand{\theenumi}{\alph{enumi}}
%\renewcommand{\labelenumi}{\theenumi)}

% Перечисление строчными буквами русского алфавита (ГОСТ 2.105-95, 4.1.7)
\makeatletter
\AddEnumerateCounter{\asbuk}{\russian@alph}{щ}      % Управляем списками/перечислениями через пакет enumitem, а он 'не знает' про asbuk, потому 'учим' его
\makeatother
%\renewcommand{\theenumi}{\asbuk{enumi}} %первый уровень нумерации
%\renewcommand{\labelenumi}{\theenumi)} %первый уровень нумерации
\renewcommand{\theenumii}{\asbuk{enumii}} %второй уровень нумерации
\renewcommand{\labelenumii}{\theenumii)} %второй уровень нумерации
\renewcommand{\theenumiii}{\arabic{enumiii}} %третий уровень нумерации
\renewcommand{\labelenumiii}{\theenumiii)} %третий уровень нумерации

\setlist{nosep,%                                    % Единый стиль для всех списков (пакет enumitem), без дополнительных интервалов.
    labelindent=\parindent,leftmargin=*%            % Каждый пункт, подпункт и перечисление записывают с абзацного отступа (ГОСТ 2.105-95, 4.1.8)
}

%%% Правильная нумерация приложений, рисунков и формул %%%
%% По ГОСТ 2.105, п. 4.3.8 Приложения обозначают заглавными буквами русского алфавита,
%% начиная с А, за исключением букв Ё, З, Й, О, Ч, Ь, Ы, Ъ.
%% Здесь также переделаны все нумерации русскими буквами.
\ifxetexorluatex
    \makeatletter
    \def\russian@Alph#1{\ifcase#1\or
       А\or Б\or В\or Г\or Д\or Е\or Ж\or
       И\or К\or Л\or М\or Н\or
       П\or Р\or С\or Т\or У\or Ф\or Х\or
       Ц\or Ш\or Щ\or Э\or Ю\or Я\else\xpg@ill@value{#1}{russian@Alph}\fi}
    \def\russian@alph#1{\ifcase#1\or
       а\or б\or в\or г\or д\or е\or ж\or
       и\or к\or л\or м\or н\or
       п\or р\or с\or т\or у\or ф\or х\or
       ц\or ш\or щ\or э\or ю\or я\else\xpg@ill@value{#1}{russian@alph}\fi}
    \def\cyr@Alph#1{\ifcase#1\or
        А\or Б\or В\or Г\or Д\or Е\or Ж\or
        И\or К\or Л\or М\or Н\or
        П\or Р\or С\or Т\or У\or Ф\or Х\or
        Ц\or Ш\or Щ\or Э\or Ю\or Я\else\xpg@ill@value{#1}{cyr@Alph}\fi}
    \def\cyr@alph#1{\ifcase#1\or
        а\or б\or в\or г\or д\or е\or ж\or
        и\or к\or л\or м\or н\or
        п\or р\or с\or т\or у\or ф\or х\or
        ц\or ш\or щ\or э\or ю\or я\else\xpg@ill@value{#1}{cyr@alph}\fi}
    \makeatother
\else
    \makeatletter
    \if@uni@ode
      \def\russian@Alph#1{\ifcase#1\or
        А\or Б\or В\or Г\or Д\or Е\or Ж\or
        И\or К\or Л\or М\or Н\or
        П\or Р\or С\or Т\or У\or Ф\or Х\or
        Ц\or Ш\or Щ\or Э\or Ю\or Я\else\@ctrerr\fi}
    \else
      \def\russian@Alph#1{\ifcase#1\or
        \CYRA\or\CYRB\or\CYRV\or\CYRG\or\CYRD\or\CYRE\or\CYRZH\or
        \CYRI\or\CYRK\or\CYRL\or\CYRM\or\CYRN\or
        \CYRP\or\CYRR\or\CYRS\or\CYRT\or\CYRU\or\CYRF\or\CYRH\or
        \CYRC\or\CYRSH\or\CYRSHCH\or\CYREREV\or\CYRYU\or
        \CYRYA\else\@ctrerr\fi}
    \fi
    \if@uni@ode
      \def\russian@alph#1{\ifcase#1\or
        а\or б\or в\or г\or д\or е\or ж\or
        и\or к\or л\or м\or н\or
        п\or р\or с\or т\or у\or ф\or х\or
        ц\or ш\or щ\or э\or ю\or я\else\@ctrerr\fi}
    \else
      \def\russian@alph#1{\ifcase#1\or
        \cyra\or\cyrb\or\cyrv\or\cyrg\or\cyrd\or\cyre\or\cyrzh\or
        \cyri\or\cyrk\or\cyrl\or\cyrm\or\cyrn\or
        \cyrp\or\cyrr\or\cyrs\or\cyrt\or\cyru\or\cyrf\or\cyrh\or
        \cyrc\or\cyrsh\or\cyrshch\or\cyrerev\or\cyryu\or
        \cyrya\else\@ctrerr\fi}
    \fi
    \makeatother
\fi


%%http://www.linux.org.ru/forum/general/6993203#comment-6994589 (используется totcount)
\makeatletter
\def\formtotal#1#2#3#4#5{%
    \newcount\@c
    \@c\totvalue{#1}\relax
    \newcount\@last
    \newcount\@pnul
    \@last\@c\relax
    \divide\@last 10
    \@pnul\@last\relax
    \divide\@pnul 10
    \multiply\@pnul-10
    \advance\@pnul\@last
    \multiply\@last-10
    \advance\@last\@c
    #2%
    \ifnum\@pnul=1#5\else%
    \ifcase\@last#5\or#3\or#4\or#4\or#4\else#5\fi
    \fi
}
\makeatother

\newcommand{\formbytotal}[5]{\total{#1}~\formtotal{#1}{#2}{#3}{#4}{#5}}

%%% Команды рецензирования %%%
\ifboolexpr{ (test {\ifnumequal{\value{draft}}{1}}) or (test {\ifnumequal{\value{showmarkup}}{1}})}{
        \newrobustcmd{\todo}[1]{\textcolor{red}{#1}}
        \newrobustcmd{\note}[2][]{\ifstrempty{#1}{#2}{\textcolor{#1}{#2}}}
        \newenvironment{commentbox}[1][]%
        {\ifstrempty{#1}{}{\color{#1}}}%
        {}
}{
        \newrobustcmd{\todo}[1]{}
        \newrobustcmd{\note}[2][]{}
        \excludecomment{commentbox}
}



%%% Переопределение именований, если иначе не сработает %%%
\gappto\captionsrussian{
	   \renewcommand{\chaptername}{ГЛАВА}
	   \renewcommand{\appendixname}{ПРИЛОЖЕНИЕ} % (ГОСТ Р 7.0.11-2011, 5.7)
	}

%%% Изображения %%%
\graphicspath{{images/}{Dissertation/images/}}         % Пути к изображениям

%%% Интервалы %%%
%% По ГОСТ Р 7.0.11-2011, пункту 5.3.6 требуется полуторный интервал
%% Реализация средствами класса (на основе setspace) ближе к типографской классике.
%% И правит сразу и в таблицах (если со звёздочкой)
%\DoubleSpacing*     % Двойной интервал
% \OnehalfSpacing*    % Полуторный интервал
\setSpacing{1.42}   % Полуторный интервал, подобный Ворду (возможно, стоит включать вместе с предыдущей строкой)

%%% Макет страницы %%%
% Выставляем значения полей (ГОСТ 7.0.11-2011, 5.3.7)
\geometry{a4paper, top=2cm, bottom=2cm, left=3cm, right=1cm, nofoot, nomarginpar} %, heightrounded, showframe
\setlength{\topskip}{0pt}   %размер дополнительного верхнего поля
\setlength{\footskip}{12.3pt} % снимет warning, согласно https://tex.stackexchange.com/a/334346

%%% Выравнивание и переносы %%%
%% http://tex.stackexchange.com/questions/241343/what-is-the-meaning-of-fussy-sloppy-emergencystretch-tolerance-hbadness
%% http://www.latex-community.org/forum/viewtopic.php?p=70342#p70342
\tolerance 1414
\hbadness 1414
\emergencystretch 1.5em % В случае проблем регулировать в первую очередь
\hfuzz 0.3pt
\vfuzz \hfuzz
%\raggedbottom
%\sloppy                 % Избавляемся от переполнений
\clubpenalty=10000      % Запрещаем разрыв страницы после первой строки абзаца
\widowpenalty=10000     % Запрещаем разрыв страницы после последней строки абзаца
\brokenpenalty=4991     % Ограничение на разрыв страницы, если строка заканчивается переносом

%%% Блок управления параметрами для выравнивания заголовков в тексте %%%
\newlength{\otstuplen}
\setlength{\otstuplen}{\theotstup\parindent}
\ifnumequal{\value{headingalign}}{0}{% выравнивание заголовков в тексте
	\newcommand{\hdngalign}{\centering}                % по центру
	\newcommand{\hdngaligni}{}% по центру
	\setlength{\otstuplen}{0pt}
}{%
	\newcommand{\hdngalign}{}                 % по левому краю
	\newcommand{\hdngaligni}{\hspace{\otstuplen}}      % по левому краю
} % В обоих случаях вроде бы без переноса, как и надо (ГОСТ Р 7.0.11-2011, 5.3.5)

%%% Оглавление %%%
\renewcommand{\cftchapterdotsep}{\cftdotsep}                % отбивка точками до номера страницы начала главы/раздела

%% Переносить слова в заголовке не допускается (ГОСТ Р 7.0.11-2011, 5.3.5). Заголовки в оглавлении должны точно повторять заголовки в тексте (ГОСТ Р 7.0.11-2011, 5.2.3). Прямого указания на запрет переносов в оглавлении нет, но по той же логике невнесения искажений в смысл, лучше в оглавлении не переносить:
\setrmarg{2.55em plus1fil}                             %To have the (sectional) titles in the ToC, etc., typeset ragged right with no hyphenation
\renewcommand{\cftchapterpagefont}{\normalfont}        % нежирные номера страниц у глав в оглавлении
\renewcommand{\cftchapterleader}{\cftdotfill{\cftchapterdotsep}}% нежирные точки до номеров страниц у глав в оглавлении
%\renewcommand{\cftchapterfont}{}                       % нежирные названия глав в оглавлении

\ifnumgreater{\value{headingdelim}}{0}{%
	\renewcommand\cftchapteraftersnum{.\space}       % добавляет точку с пробелом после номера раздела в оглавлении
}{}
\ifnumgreater{\value{headingdelim}}{1}{%
	\renewcommand\cftsectionaftersnum{.\space}       % добавляет точку с пробелом после номера подраздела в оглавлении
	\renewcommand\cftsubsectionaftersnum{.\space}    % добавляет точку с пробелом после номера подподраздела в оглавлении
	\renewcommand\cftsubsubsectionaftersnum{.\space} % добавляет точку с пробелом после номера подподподраздела в оглавлении
	\AfterEndPreamble{% без этого polyglossia сама всё переопределяет
		\setsecnumformat{\csname the#1\endcsname.\space}
	}
}{%
	\AfterEndPreamble{% без этого polyglossia сама всё переопределяет
		\setsecnumformat{\csname the#1\endcsname\quad}
	}
}

\renewcommand*{\cftappendixname}{\appendixname\space} % Слово Приложение в оглавлении

%%% Колонтитулы %%%
% Порядковый номер страницы печатают на середине верхнего поля страницы (ГОСТ Р 7.0.11-2011, 5.3.8)
% \makeevenhead{plain}{}{\rmfamily\thepage}{}
% \makeoddhead{plain}{}{\rmfamily\thepage}{}
% \makeevenfoot{plain}{}{}{}
% \makeoddfoot{plain}{}{}{}
\makeevenhead{plain}{}{}{}
\makeoddhead{plain}{}{}{}
% \makeevenfoot{plain}{}{\rmfamily\thepage}{}
% \makeoddfoot{plain}{}{\rmfamily\thepage}{}
\makeevenfoot{plain}{}{\small\thepage}{}
\makeoddfoot{plain}{}{\small\thepage}{}
\pagestyle{plain}

%%% добавить Стр. над номерами страниц в оглавлении
%%% http://tex.stackexchange.com/a/306950
\newif\ifendTOC

% \newcommand*{\tocheader}{
% 	\ifnumequal{\value{pgnum}}{1}{%
% 		\ifendTOC\else\hbox to \linewidth%
% 		{\noindent{}~\hfill{Стр.}}\par%
% 		\ifnumless{\value{page}}{3}{}{%
% 			\vspace{0.5\onelineskip}
% 		}
% 		\afterpage{\tocheader}
% 		\fi%
% 	}{}%
% }%

%%% Оформление заголовков глав, разделов, подразделов %%%
%% Работа должна быть выполнена ... размером шрифта 12-14 пунктов (ГОСТ Р 7.0.11-2011, 5.3.8). То есть не должно быть надписей шрифтом более 14. Так и поставим.
%% Эти установки будут давать одинаковый результат независимо от выбора базовым шрифтом 12 пт или 14 пт
\newcommand{\basegostsectionfont}{\fontsize{14pt}{16pt}\selectfont\bfseries}

\makechapterstyle{thesisgost}{%
	\chapterstyle{default}
	\setlength{\beforechapskip}{0pt}
	\setlength{\midchapskip}{0pt}
	\setlength{\afterchapskip}{\theintvl\curtextsize}
	% \renewcommand*{\chapnamefont}{\basegostsectionfont}
	% \renewcommand*{\chapnumfont}{\basegostsectionfont}
	% \renewcommand*{\chaptitlefont}{\basegostsectionfont}
	\renewcommand*{\chapnamefont}{\normalfont}
	\renewcommand*{\chapnumfont}{\normalfont}
	\renewcommand*{\chaptitlefont}{\normalfont}
	\renewcommand*{\chapterheadstart}{}
	\ifnumgreater{\value{headingdelim}}{0}{%
		\renewcommand*{\afterchapternum}{.\space}   % добавляет точку с пробелом после номера раздела
	}{%
		\renewcommand*{\afterchapternum}{\quad}     % добавляет \quad после номера раздела
	}
	\renewcommand*{\printchapternum}{\hdngalign\chapnumfont \thechapter}
	\renewcommand*{\printchaptername}{}
	\renewcommand*{\printchapternonum}{\hdngalign}
}

\makeatletter
\makechapterstyle{thesisgostchapname}{%
	\chapterstyle{thesisgost}
	\renewcommand*{\printchapternum}{\chapnumfont \thechapter}
	\renewcommand*{\printchaptername}{\hdngalign\chapnamefont \@chapapp} %
}
\makeatother

\chapterstyle{thesisgost}

% \setsecheadstyle{\basegostsectionfont\hdngalign}
\setsecheadstyle{\basegostsectionfont\hspace{\otstuplen}}
% \setsecindent{\otstuplen}

% \setsubsecheadstyle{\basegostsectionfont\hdngalign}
\setsubsecheadstyle{\basegostsectionfont\hspace{\otstuplen}}
% \setsubsecindent{\otstuplen}

% \setsubsubsecheadstyle{\basegostsectionfont\hdngalign}
\setsubsubsecheadstyle{\basegostsectionfont\hspace{\otstuplen}}
% \setsubsubsecindent{\otstuplen}

% \sethangfrom{\noindent #1} %все заголовки подразделов центрируются с учетом номера, как block

\ifnumequal{\value{chapstyle}}{1}{%
	\chapterstyle{thesisgostchapname}
	\renewcommand*{\cftchaptername}{\chaptername\space} % будет вписано слово Глава перед каждым номером раздела в оглавлении
}{}%

%%% Интервалы между заголовками
% \setbeforesecskip{\theintvl\curtextsize}% Заголовки отделяют от текста сверху и снизу тремя интервалами (ГОСТ Р 7.0.11-2011, 5.3.5).
% \setaftersecskip{\theintvl\curtextsize}
% \setbeforesubsecskip{\theintvl\curtextsize}
% \setaftersubsecskip{\theintvl\curtextsize}
% \setbeforesubsubsecskip{\theintvl\curtextsize}
% \setaftersubsubsecskip{\theintvl\curtextsize}

%%% Вертикальные интервалы глав (\chapter) в оглавлении как и у заголовков
% раскомментировать следующие 2
% \setlength{\cftbeforechapterskip}{0pt plus 0pt}   % ИЛИ эти 2 строки из учебника
% \renewcommand*{\insertchapterspace}{}
% или эту
% \renewcommand*{\cftbeforechapterskip}{0em}


%%% Блок дополнительного управления размерами заголовков
\ifnumequal{\value{headingsize}}{1}{% Пропорциональные заголовки и базовый шрифт 14 пт
	% \renewcommand{\basegostsectionfont}{\large\bfseries}
	% \renewcommand*{\chapnamefont}{\Large\bfseries}
	% \renewcommand*{\chapnumfont}{\Large\bfseries}
	% \renewcommand*{\chaptitlefont}{\Large\bfseries}
	\renewcommand{\basegostsectionfont}{\large\bfseries}
	\renewcommand*{\chapnamefont}{\Large\bfseries}
	\renewcommand*{\chapnumfont}{\Large\bfseries}
	\renewcommand*{\chaptitlefont}{\Large\bfseries}
}{}

%%% Счётчики %%%

%% Упрощённые настройки шаблона диссертации: нумерация формул, таблиц, рисунков
\ifnumequal{\value{contnumeq}}{1}{%
	\counterwithout{equation}{chapter} % Убираем связанность номера формулы с номером главы/раздела
}{}
\ifnumequal{\value{contnumfig}}{1}{%
	\counterwithout{figure}{chapter}   % Убираем связанность номера рисунка с номером главы/раздела
}{}
\ifnumequal{\value{contnumtab}}{1}{%
	\counterwithout{table}{chapter}    % Убираем связанность номера таблицы с номером главы/раздела
}{}

\AfterEndPreamble{
	%% регистрируем счётчики в системе totcounter
	\regtotcounter{totalcount@figure}
	\regtotcounter{totalcount@lstlisting}
	\regtotcounter{totalcount@table}       % Если иным способом поставить в преамбуле то ошибка в числе таблиц
	\regtotcounter{TotPages}               % Если иным способом поставить в преамбуле то ошибка в числе страниц
	\newtotcounter{TotPagesNoAppendix} 
	\newtotcounter{totalappendix}
	\newtotcounter{totalchapter}
}


% для вертикального центрирования ячеек в tabulary
\def\zz{\ifx\[$\else\aftergroup\zzz\fi}
%$ \] % <-- чиним подсветку синтаксиса в некоторых редакторах
\def\zzz{\setbox0\lastbox
	\dimen0\dimexpr\extrarowheight + \ht0-\dp0\relax
	\setbox0\hbox{\raise-.5\dimen0\box0}%
	\ht0=\dimexpr\ht0+\extrarowheight\relax
	\dp0=\dimexpr\dp0+\extrarowheight\relax
	\box0
}

\lstdefinelanguage{Renhanced}%
{keywords={abbreviate,abline,abs,acos,acosh,action,add1,add,%
		aggregate,alias,Alias,alist,all,anova,any,aov,aperm,append,apply,%
		approx,approxfun,apropos,Arg,args,array,arrows,as,asin,asinh,%
		atan,atan2,atanh,attach,attr,attributes,autoload,autoloader,ave,%
		axis,backsolve,barplot,basename,besselI,besselJ,besselK,besselY,%
		beta,binomial,body,box,boxplot,break,browser,bug,builtins,bxp,by,%
		c,C,call,Call,case,cat,category,cbind,ceiling,character,char,%
		charmatch,check,chol,chol2inv,choose,chull,class,close,cm,codes,%
		coef,coefficients,co,col,colnames,colors,colours,commandArgs,%
		comment,complete,complex,conflicts,Conj,contents,contour,%
		contrasts,contr,control,helmert,contrib,convolve,cooks,coords,%
		distance,coplot,cor,cos,cosh,count,fields,cov,covratio,wt,CRAN,%
		create,crossprod,cummax,cummin,cumprod,cumsum,curve,cut,cycle,D,%
		data,dataentry,date,dbeta,dbinom,dcauchy,dchisq,de,debug,%
		debugger,Defunct,default,delay,delete,deltat,demo,de,density,%
		deparse,dependencies,Deprecated,deriv,description,detach,%
		dev2bitmap,dev,cur,deviance,off,prev,,dexp,df,dfbetas,dffits,%
		dgamma,dgeom,dget,dhyper,diag,diff,digamma,dim,dimnames,dir,%
		dirname,dlnorm,dlogis,dnbinom,dnchisq,dnorm,do,dotplot,double,%
		download,dpois,dput,drop,drop1,dsignrank,dt,dummy,dump,dunif,%
		duplicated,dweibull,dwilcox,dyn,edit,eff,effects,eigen,else,%
		emacs,end,environment,env,erase,eval,equal,evalq,example,exists,%
		exit,exp,expand,expression,External,extract,extractAIC,factor,%
		fail,family,fft,file,filled,find,fitted,fivenum,fix,floor,for,%
		For,formals,format,formatC,formula,Fortran,forwardsolve,frame,%
		frequency,ftable,ftable2table,function,gamma,Gamma,gammaCody,%
		gaussian,gc,gcinfo,gctorture,get,getenv,geterrmessage,getOption,%
		getwd,gl,glm,globalenv,gnome,GNOME,graphics,gray,grep,grey,grid,%
		gsub,hasTsp,hat,heat,help,hist,home,hsv,httpclient,I,identify,if,%
		ifelse,Im,image,\%in\%,index,influence,measures,inherits,install,%
		installed,integer,interaction,interactive,Internal,intersect,%
		inverse,invisible,IQR,is,jitter,kappa,kronecker,labels,lapply,%
		layout,lbeta,lchoose,lcm,legend,length,levels,lgamma,library,%
		licence,license,lines,list,lm,load,local,locator,log,log10,log1p,%
		log2,logical,loglin,lower,lowess,ls,lsfit,lsf,ls,machine,Machine,%
		mad,mahalanobis,make,link,margin,match,Math,matlines,mat,matplot,%
		matpoints,matrix,max,mean,median,memory,menu,merge,methods,min,%
		missing,Mod,mode,model,response,mosaicplot,mtext,mvfft,na,nan,%
		names,omit,nargs,nchar,ncol,NCOL,new,next,NextMethod,nextn,%
		nlevels,nlm,noquote,NotYetImplemented,NotYetUsed,nrow,NROW,null,%
		numeric,\%o\%,objects,offset,old,on,Ops,optim,optimise,optimize,%
		options,or,order,ordered,outer,package,packages,page,pairlist,%
		pairs,palette,panel,par,parent,parse,paste,path,pbeta,pbinom,%
		pcauchy,pchisq,pentagamma,persp,pexp,pf,pgamma,pgeom,phyper,pico,%
		pictex,piechart,Platform,plnorm,plogis,plot,pmatch,pmax,pmin,%
		pnbinom,pnchisq,pnorm,points,poisson,poly,polygon,polyroot,pos,%
		postscript,power,ppoints,ppois,predict,preplot,pretty,Primitive,%
		print,prmatrix,proc,prod,profile,proj,prompt,prop,provide,%
		psignrank,ps,pt,ptukey,punif,pweibull,pwilcox,q,qbeta,qbinom,%
		qcauchy,qchisq,qexp,qf,qgamma,qgeom,qhyper,qlnorm,qlogis,qnbinom,%
		qnchisq,qnorm,qpois,qqline,qqnorm,qqplot,qr,Q,qty,qy,qsignrank,%
		qt,qtukey,quantile,quasi,quit,qunif,quote,qweibull,qwilcox,%
		rainbow,range,rank,rbeta,rbind,rbinom,rcauchy,rchisq,Re,read,csv,%
		csv2,fwf,readline,socket,real,Recall,rect,reformulate,regexpr,%
		relevel,remove,rep,repeat,replace,replications,report,require,%
		resid,residuals,restart,return,rev,rexp,rf,rgamma,rgb,rgeom,R,%
		rhyper,rle,rlnorm,rlogis,rm,rnbinom,RNGkind,rnorm,round,row,%
		rownames,rowsum,rpois,rsignrank,rstandard,rstudent,rt,rug,runif,%
		rweibull,rwilcox,sample,sapply,save,scale,scan,scan,screen,sd,se,%
		search,searchpaths,segments,seq,sequence,setdiff,setequal,set,%
		setwd,show,sign,signif,sin,single,sinh,sink,solve,sort,source,%
		spline,splinefun,split,sqrt,stars,start,stat,stem,step,stop,%
		storage,strstrheight,stripplot,strsplit,structure,strwidth,sub,%
		subset,substitute,substr,substring,sum,summary,sunflowerplot,svd,%
		sweep,switch,symbol,symbols,symnum,sys,status,system,t,table,%
		tabulate,tan,tanh,tapply,tempfile,terms,terrain,tetragamma,text,%
		time,title,topo,trace,traceback,transform,tri,trigamma,trunc,try,%
		ts,tsp,typeof,unclass,undebug,undoc,union,unique,uniroot,unix,%
		unlink,unlist,unname,untrace,update,upper,url,UseMethod,var,%
		variable,vector,Version,vi,warning,warnings,weighted,weights,%
		which,while,window,write,\%x\%,x11,X11,xedit,xemacs,xinch,xor,%
		xpdrows,xy,xyinch,yinch,zapsmall,zip},%
	otherkeywords={!,!=,~,$,*,\%,\&,\%/\%,\%*\%,\%\%,<-,<<-},%$
	alsoother={._$},%$
	sensitive,%
	morecomment=[l]\#,%
	morestring=[d]",%
	morestring=[d]'% 2001 Robert Denham
}%

%решаем проблему с кириллицей в комментариях (в pdflatex) https://tex.stackexchange.com/a/103712
\lstset{extendedchars=true,keepspaces=true,literate={Ö}{{\"O}}1
	{Ä}{{\"A}}1
	{Ü}{{\"U}}1
	{ß}{{\ss}}1
	{ü}{{\"u}}1
	{ä}{{\"a}}1
	{ö}{{\"o}}1
	{~}{{\textasciitilde}}1
	{а}{{\selectfont\char224}}1
	{б}{{\selectfont\char225}}1
	{в}{{\selectfont\char226}}1
	{г}{{\selectfont\char227}}1
	{д}{{\selectfont\char228}}1
	{е}{{\selectfont\char229}}1
	{ё}{{\"e}}1
	{ж}{{\selectfont\char230}}1
	{з}{{\selectfont\char231}}1
	{и}{{\selectfont\char232}}1
	{й}{{\selectfont\char233}}1
	{к}{{\selectfont\char234}}1
	{л}{{\selectfont\char235}}1
	{м}{{\selectfont\char236}}1
	{н}{{\selectfont\char237}}1
	{о}{{\selectfont\char238}}1
	{п}{{\selectfont\char239}}1
	{р}{{\selectfont\char240}}1
	{с}{{\selectfont\char241}}1
	{т}{{\selectfont\char242}}1
	{у}{{\selectfont\char243}}1
	{ф}{{\selectfont\char244}}1
	{х}{{\selectfont\char245}}1
	{ц}{{\selectfont\char246}}1
	{ч}{{\selectfont\char247}}1
	{ш}{{\selectfont\char248}}1
	{щ}{{\selectfont\char249}}1
	{ъ}{{\selectfont\char250}}1
	{ы}{{\selectfont\char251}}1
	{ь}{{\selectfont\char252}}1
	{э}{{\selectfont\char253}}1
	{ю}{{\selectfont\char254}}1
	{я}{{\selectfont\char255}}1
	{А}{{\selectfont\char192}}1
	{Б}{{\selectfont\char193}}1
	{В}{{\selectfont\char194}}1
	{Г}{{\selectfont\char195}}1
	{Д}{{\selectfont\char196}}1
	{Е}{{\selectfont\char197}}1
	{Ё}{{\"E}}1
	{Ж}{{\selectfont\char198}}1
	{З}{{\selectfont\char199}}1
	{И}{{\selectfont\char200}}1
	{Й}{{\selectfont\char201}}1
	{К}{{\selectfont\char202}}1
	{Л}{{\selectfont\char203}}1
	{М}{{\selectfont\char204}}1
	{Н}{{\selectfont\char205}}1
	{О}{{\selectfont\char206}}1
	{П}{{\selectfont\char207}}1
	{Р}{{\selectfont\char208}}1
	{С}{{\selectfont\char209}}1
	{Т}{{\selectfont\char210}}1
	{У}{{\selectfont\char211}}1
	{Ф}{{\selectfont\char212}}1
	{Х}{{\selectfont\char213}}1
	{Ц}{{\selectfont\char214}}1
	{Ч}{{\selectfont\char215}}1
	{Ш}{{\selectfont\char216}}1
	{Щ}{{\selectfont\char217}}1
	{Ъ}{{\selectfont\char218}}1
	{Ы}{{\selectfont\char219}}1
	{Ь}{{\selectfont\char220}}1
	{Э}{{\selectfont\char221}}1
	{Ю}{{\selectfont\char222}}1
	{Я}{{\selectfont\char223}}1
	{і}{{\selectfont\char105}}1
	{ї}{{\selectfont\char168}}1
	{є}{{\selectfont\char185}}1
	{ґ}{{\selectfont\char160}}1
	{І}{{\selectfont\char73}}1
	{Ї}{{\selectfont\char136}}1
	{Є}{{\selectfont\char153}}1
	{Ґ}{{\selectfont\char128}}1
}

% Ширина текста минус ширина надписи 999
\newlength{\twless}
\newlength{\lmarg}
\setlength{\lmarg}{\widthof{999}}   % ширина надписи 999
\setlength{\twless}{\textwidth-\lmarg}

\lstset{ %
	%    language=R,                     %  Язык указать здесь, если во всех листингах преимущественно один язык, в результате часть настроек может пойти только для этого языка
	numbers=left,                   % where to put the line-numbers
	numberstyle=\fontsize{12pt}{14pt}\selectfont\color{Gray},  % the style that is used for the line-numbers
	firstnumber=1,                  % в этой и следующей строках задаётся поведение нумерации 5, 10, 15...
	stepnumber=1,                   % the step between two line-numbers. If it's 1, each line will be numbered
	numbersep=5pt,                  % how far the line-numbers are from the code
	backgroundcolor=\color{white},  % choose the background color. You must add \usepackage{color}
	showspaces=false,               % show spaces adding particular underscores
	showstringspaces=false,         % underline spaces within strings
	showtabs=false,                 % show tabs within strings adding particular underscores
	frame=leftline,                 % adds a frame of different types around the code
	rulecolor=\color{black},        % if not set, the frame-color may be changed on line-breaks within not-black text (e.g. commens (green here))
	tabsize=4,                      % sets default tabsize to 2 spaces
	captionpos=b,                   % sets the caption-position to top
	breaklines=true,                % sets automatic line breaking
	breakatwhitespace=false,        % sets if automatic breaks should only happen at whitespace
	%    title=\lstname,                 % show the filename of files included with \lstinputlisting;
	% also try caption instead of title
	basicstyle=\fontsize{12pt}{14pt}\selectfont\ttfamily,% the size of the fonts that are used for the code
	%    keywordstyle=\color{blue},      % keyword style
	commentstyle=\color{ForestGreen}\emph,% comment style
	stringstyle=\color{Mahogany},   % string literal style
	escapeinside={\%*}{*)},         % if you want to add a comment within your code
	morekeywords={*,...},           % if you want to add more keywords to the set
	inputencoding=utf8,             % кодировка кода
	xleftmargin={\lmarg},           % Чтобы весь код и полоска с номерами строк была смещена влево, так чтобы цифры не вылезали за пределы текста слева
}

%http://tex.stackexchange.com/questions/26872/smaller-frame-with-listings
% Окружение, чтобы листинг был компактнее обведен рамкой, если она задается, а не на всю ширину текста
\makeatletter
\newenvironment{SmallListing}[1][]
{\lstset{#1}\VerbatimEnvironment\begin{VerbatimOut}{VerbEnv.tmp}}
	{\end{VerbatimOut}\settowidth\@tempdima{%
		\lstinputlisting{VerbEnv.tmp}}
	\minipage{\@tempdima}\lstinputlisting{VerbEnv.tmp}\endminipage}
\makeatother

\DefineVerbatimEnvironment% с шрифтом 12 пт
{Verb}{Verbatim}
{fontsize=\fontsize{12pt}{14pt}\selectfont}

\newfloat[chapter]{ListingEnv}{lol}{Листинг}

\renewcommand{\lstlistingname}{Листинг}

%Общие счётчики окружений листингов
%http://tex.stackexchange.com/questions/145546/how-to-make-figure-and-listing-share-their-counter
% Если смешивать плавающие и не плавающие окружения, то могут быть проблемы с нумерацией
\makeatletter
\AfterEndPreamble{% https://tex.stackexchange.com/a/252682
	\let\c@ListingEnv\relax % drop existing counter "ListingEnv"
	\newaliascnt{ListingEnv}{lstlisting} % команда требует пакет aliascnt
	\let\ftype@lstlisting\ftype@ListingEnv % give the floats the same precedence
}
\makeatother

% значок С++ — используйте команду \cpp
\newcommand{\cpp}{%
	C\nolinebreak\hspace{-.05em}%
	\raisebox{.2ex}{+}\nolinebreak\hspace{-.10em}%
	\raisebox{.2ex}{+}%
}

%%%  Чересстрочное форматирование таблиц
%% http://tex.stackexchange.com/questions/278362/apply-italic-formatting-to-every-other-row
\newcounter{rowcnt}
\newcommand\altshape{\ifnumodd{\value{rowcnt}}{\color{red}}{\vspace*{-1ex}\itshape}}
% \AtBeginEnvironment{tabular}{\setcounter{rowcnt}{1}}
% \AtEndEnvironment{tabular}{\setcounter{rowcnt}{0}}

%%% Ради примера во второй главе
\let\originalepsilon\epsilon
\let\originalphi\phi
\let\originalkappa\kappa
\let\originalle\le
\let\originalleq\leq
\let\originalge\ge
\let\originalgeq\geq
\let\originalemptyset\emptyset
\let\originaltan\tan
\let\originalcot\cot
\let\originalcsc\csc

%%% Русская традиция начертания математических знаков
\renewcommand{\le}{\ensuremath{\leqslant}}
\renewcommand{\leq}{\ensuremath{\leqslant}}
\renewcommand{\ge}{\ensuremath{\geqslant}}
\renewcommand{\geq}{\ensuremath{\geqslant}}
\renewcommand{\emptyset}{\varnothing}

%%% Русская традиция начертания математических функций (на случай копирования из зарубежных источников)
\renewcommand{\tan}{\operatorname{tg}}
\renewcommand{\cot}{\operatorname{ctg}}
\renewcommand{\csc}{\operatorname{cosec}}

%%% Русская традиция начертания греческих букв (греческие буквы вертикальные, через пакет upgreek)
\renewcommand{\epsilon}{\ensuremath{\upvarepsilon}}   %  русская традиция записи
\renewcommand{\phi}{\ensuremath{\upvarphi}}
%\renewcommand{\kappa}{\ensuremath{\varkappa}}
\renewcommand{\alpha}{\upalpha}
\renewcommand{\beta}{\upbeta}
\renewcommand{\gamma}{\upgamma}
\renewcommand{\delta}{\updelta}
\renewcommand{\varepsilon}{\upvarepsilon}
\renewcommand{\zeta}{\upzeta}
\renewcommand{\eta}{\upeta}
\renewcommand{\theta}{\uptheta}
\renewcommand{\vartheta}{\upvartheta}
\renewcommand{\iota}{\upiota}
\renewcommand{\kappa}{\upkappa}
\renewcommand{\lambda}{\uplambda}
\renewcommand{\mu}{\upmu}
\renewcommand{\nu}{\upnu}
\renewcommand{\xi}{\upxi}
\renewcommand{\pi}{\uppi}
\renewcommand{\varpi}{\upvarpi}
\renewcommand{\rho}{\uprho}
%\renewcommand{\varrho}{\upvarrho}
\renewcommand{\sigma}{\upsigma}
%\renewcommand{\varsigma}{\upvarsigma}
\renewcommand{\tau}{\uptau}
\renewcommand{\upsilon}{\upupsilon}
\renewcommand{\varphi}{\upvarphi}
\renewcommand{\chi}{\upchi}
\renewcommand{\psi}{\uppsi}
\renewcommand{\omega}{\upomega}
