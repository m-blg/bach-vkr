%%% Переопределение именований %%%
\renewcommand{\contentsname}{ОГЛАВЛЕНИЕ}% (ГОСТ Р 7.0.11-2011, 4)
\renewcommand{\figurename}{Рисунок}% (ГОСТ Р 7.0.11-2011, 5.3.9)
\renewcommand{\tablename}{Таблица}% (ГОСТ Р 7.0.11-2011, 5.3.10)
\renewcommand{\listfigurename}{Список рисунков}%
\renewcommand{\listtablename}{Список таблиц}%
\renewcommand{\bibname}{СПИСОК ЛИТЕРАТУРЫ}%
% Переопределения названий для nomencl. Так как опция russian не для utf8
\renewcommand{\nomname}{Список сокращений и условных обозначений}%
\renewcommand{\eqdeclaration}[1]{, см.~(#1)}%
\renewcommand{\pagedeclaration}[1]{, стр.~#1}%
\renewcommand{\nomAname}{Латинские буквы}%
\renewcommand{\nomGname}{Греческие буквы}%
\renewcommand{\nomXname}{Верхние индексы}%
\renewcommand{\nomZname}{Индексы}%