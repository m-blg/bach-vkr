\chapter*{РЕФЕРАТ}
% \addcontentsline{toc}{chapter}{РЕФЕРАТ} 

\begin{center}
Работа \formbytotal{TotPagesNoAppendix}{страниц}{у}{ы}{}, 
    \formbytotal{totalcount@figure}{рисун}{ок}{ка}{ков},
    % \formbytotal{totalcount@lstlisting}{листинг}{}{а}{ов},
    \formbytotal{totalappendix}{приложен}{ие}{ия}{ий},
    \formbytotal{citenum}{источник}{}{а}{ов}.
\end{center}

    КОМПИЛЯТОРЫ, СИ, СИНТАКСИЧЕСКИЕ ДЕРЕВЬЯ, ГРАММАТИЧЕСКИЙ РАЗБОР, ПАРСИНГ, ЛЕКСИНГ, МАКРОСЫ, МЕТАПРОГРАММИРОВАНИЕ, ЯЗЫКИ ПРОГРАММИРОВАНИЯ %\\[2pt]

    \vspace{5mm}
% \paragraph*{Cтруктура работы.}
Данная работа посвящена разработке библиотеки грамматического разбора языка Си и разработке прототипа компилятора\break языка \textquote{Extended C} (\textquote{Расширенный Си}, далее \textquote{EC}) на основе этой библиотеки.

Объектом разработки является source-to-source компилятор \verb|ecc|, языка EC, являющейся небольшой надстройкой над языком Си.
Целью работы является совершенствование инструментария по работе с языком Си.

Задачами ВКР являются обзор и анализ существующих систем метапрограммирования в других языках, 
разработка аналогичной системы для языка EC,
демонстрация практической выгоды данной системы.

Новизна данной работы заключается в усовершенствовании устоявшегося языка Си.

Экономическая эффективность работы заключается в том, что реализованная в данной работе подсистема метапрограммирования позволяет, в некоторых случаях, значительно сократить работу программиста. %, что продемонстрировано[\ref{use-ex}]. % TODO ref

% \paragraph*{Структура работы.} 
Основная часть отчета состоит из двух глав:\\
1. обзор архитектуры проекта, примеры использования, сравнение его с решениями в других языках\\
2. рассмотрение деталей и алгоритмов работы компилятора

% \clearpage

% \begin{lstlisting}[language=c]
% @post_include("demo.h")

% @derive(DebugFormat)
% struct Foo {
%     int x;
%     int y;
%     int z;
% };

% int main() {
%     ctx_init_default();
%     print();
%     ctx_deinit();
% }

% void 
% print() {
%     Foo foo = (Foo) {
%         .x = 3,
%         .y = 4,
%         .z = 5,
%     };
%     dbgp(Foo, &foo);

% }
% \end{lstlisting}

% \clearpage
% \begin{lstlisting}[language=bash]
% $ ecc -p macro.ec example.ec -o example
% \end{lstlisting}

% \clearpage

% \begin{lstlisting}[language=bash]
% $ ./example
% Foo:
%     x: 3
%     y: 4
%     z: 5
% \end{lstlisting}

%% на случай ошибок оставляю исходный кусок на месте, закомментированным
%Полный объём диссертации составляет  \ref*{TotPages}~страницу
%с~\totalfigures{}~рисунками и~\totaltables{}~таблицами. Список литературы
%содержит \total{citenum}~наименований.
%
% Полный объём диссертации составляет
% \formbytotal{TotPages}{страниц}{у}{ы}{}, включая
% \formbytotal{totalcount@figure}{рисун}{ок}{ка}{ков} и
% \formbytotal{totalcount@table}{таблиц}{у}{ы}{}.
% Список литературы содержит
% \formbytotal{citenum}{наименован}{ие}{ия}{ий}.




% \newpage
% \section*{Основное содержание работы}

% В Главе~\ref{ch:ch1}... \pageref{extras:c_ast}

% \section*{Публикации автора по теме диссертации}


% Основные результаты по теме диссертации изложены в \theAllMyPapers~публикациях. 
% Из них
% %4 изданы в журналах, рекомендованных ВАК, 
% \theScopusPapers~опубликовано в изданиях, индексируемых в базе цитирования Scopus. 
% %Также имеется 1 свидетельство о государственной регистрации программ для ЭВМ.

% В международных изданиях, индексируемых в базе данных Scopus:
% \begin{refsection}[biblio/own.bib]
% \nocite{*}
% \printbibliography[
%     keyword=scopus,
%     %title={В международных изданиях, индексируемых в базе данных Scopus}, 
%     %heading=subbibliography,
%     heading=none,
%     resetnumbers=true
% ]
% \end{refsection}



% В международных изданиях, индексируемых в базе данных Web of Science:
% \begin{refsection}[biblio/own.bib]
% \nocite{*}
% \printbibliography[
%     keyword=wos,
%     %title={В международных изданиях, индексируемых в базе данных Web of Science}, 
%     %heading=subbibliography,
%     heading=none,
%     resetnumbers=true
% ]
% \end{refsection}
% Список всех публикаций автора по теме диссертации:
% \begin{refsection}[biblio/own.bib]
% \nocite{*}
% \printbibliography[
%     keyword=own,
%     %title={Список всех публикаций автора по теме диссертации}, 
%     %heading=subbibliography,
%     heading=none,
%     resetnumbers=true
% ]
% \end{refsection}
