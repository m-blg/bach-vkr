\chapter{ДЕТАЛИ РЕАЛИЗАЦИИ}
\label{ch:ch2}

\section{Практическая часть: введение}

Разработана библиотека ec.so, в которой реализованны возможности source-to-source компилятора языка Си.
К этой библиотеке написана небольшая утилита \verb|ecc|, реализующая для нее cli интерфейс.

\subsection*{Интерфейс ecc}
Данная утилита(\verb|ecc|) является основным местом, куда пользователь идет при желании использовать компилятор.

\verb|ecc| использует библиотеку GNU argp для разбора и фильтрации аргументов, приходящих из командной строки. 
Argp автомачески генерирует сообщение[\ref{details:ecc-api:argp-usage-err}] об ошибке, при подаче неправильных аргументов.

\begin{lstlisting}[language=bash, caption={Пример сообщения об ошике, сгенерированного argp}, label={details:ecc-api:argp-usage-err}]
$ build/sandbox/ecc                                   
Usage: ecc [OPTION...] file...
Try `ecc --help' or `ecc --usage' for more information.
\end{lstlisting}

Так же argp добавляет ключ \verb|--help| и автоматически генерирует сообщение[\ref{details:ecc-api:argp-help}] со всеми основными ключами.
\begin{lstlisting}[language=bash, caption={Пример сообщения об использовании, сгенерированного argp}, label={details:ecc-api:argp-help}]
$ build/sandbox/ecc --help
Usage: ecc [OPTION...] file...
ecc - Extended C Compiler. Source-to-source compiler.

  -o, --output=FILE          Output to FILE instead of standard output
  -p, --proc-macro-file-path=<path>
                             Specify procedural macro file path
  -v, --verbose              Produce verbose output
  -?, --help                 Give this help list
      --usage                Give a short usage message
  -V, --version              Print program version

Mandatory or optional arguments to long options are also mandatory or optional
for any corresponding short options.
\end{lstlisting}

В случае успешного получения аргументов от \verb|arpg| \verb|ecc| далее просто вызывает соответвующие аргументам функции библиотеки \verb|ec|.


\subsection*{Структура компиляция языка}
Компиляция языка EC организована в виде последовательности проходов(passes), которые шаг за шагом преобразуют исходный текст программы в AST, 
трансформируют его, применяя макро-функции, переупорядочевая его, и транслируют в исходный код языка Си. 
Таким образом данный компилятор является sorce-to-source компилятором, т.к. преобразует исходный код одного языка к коду другого.

Проходы можно разделить на три группы:
\begin{itemize}
    \item Десериализирующие: к данной группе относятся проходы выводящие структуру из линейной последовательности элементов(символы, токены). 
    Сюда относятся проходы лексического[\ref{passes:lexing}] и грамматического разборов[\ref{passes:parsing}]
    \item Анализирующие и трансформирующие: к данной группе относятся проходы выполняющие манипуляции над AST деревом. 
    Сюда относятся проходы:
    \begin{itemize}
        \item Проход применения макросов[\ref{passes:macro-expansion}]
        \item Проход упорядочевания[\ref{passes:ordering}]
    \end{itemize}
    \item Сериализирующие: к данной группе относятся проходы уплощающие AST дерево в линейную последовательность элементов.
    Сюда относятся проходы:
    \begin{itemize}
        \item Проход трансляции в Си[\ref{passes:compile-c}]
        \item Проход трансляции в graphviz[\ref{passes:compile-dot}]
    \end{itemize}
\end{itemize}

В будующем планируется, что пользователь сможет добавлять свои проходы в процесс компиляции.

Далее в данной главе каждый из проходов будет рассмотрен в деталях.







Библиотека реализованна на языке C23, используется некоторые улучшения, добавленные в новом стандарте.

Используются:
\begin{enumerate}
  \item ключевое слово \textbf{auto} для автоматического вывода типов
  \item взятие адреса у литерала, например \verb|&(int) {3}|
  \item gcc выражения-утдверждения (statement expressions) вида \verb|({x = 3; x;})|
  \item gcc \verb|,##| оператор в макросах
\end{enumerate}


\subsection{Основные примитивы}
Для написания основной библиотки используюется дочерняя библиотека[\ref{extras:c-core}] с основными примитивами: 
строковыми, ввода-вывода, аллокации памяти и др.

Данная библиотека реализована в виде единичных самодостаточных заголовочных файлов, 
которые могут быть в зависимости от флага препроцессора либо заголовочными файлами либо файлами имплементации, 
по принципу stb библиотек\cite{stb_libs}.

% \subsubsection{Макросы}

% Для определений структур и энумераций используются следующие макросы:

% \beginminteddef{c}
% #define struct_decl(name) \
% typedef struct name name; \
% struct name; \

% #define enum_decl(name) \
% typedef enum name name; \
% enum name; \

% #define struct_def(name, fields) \
% typedef struct name name; \
% struct name fields; \

% #define enum_def(name, ...) \
% typedef enum name name; \
% enum name {__VA_ARGS__}; \
% \end{minted} 


% \subsubsection{Обработка ошибок}

% Ошибки кодируются как enum, например:

% \begin{listing}
% \caption{Ошибки Аллокатора}
% \beginminteddef{c}
% enum_def(AllocatorError,
%     ALLOCATOR_ERROR_OK,
%     ALLOCATOR_ERROR_MEM_ALLOC,
%     ALLOCATOR_ERROR_COUNT
% )
% #define ALLOCATOR_ERROR(ERR) ((AllocatorError)ALLOCATOR_ERROR_##ERR)
% \end{minted}
% \end{listing}

% Значение OK кодируется как 0, остальные значения кодируются положительными числами.

% Для обработки ошибок используются следующие макросы:
% % \begin{minted}[linenos, frame=single]{c}
% \beginminteddef{c}
% #define IS_OK(e) ({ \
%     auto _r = (e); \
%     *(int *)&(_r) == 0; \
%     })
% #define IS_ERR(e) ({ \
%     auto _r = (e); \
%     *(int *)&(_r) != 0; \
%     })
% #define TRY(expr) { \
%     auto _e_ = (expr); \
%     if (IS_ERR(_e_)) { \
%         return _e_; \
%     } \
%   }
% #define ASSERT(expr) \
%     if (!(expr)) { \
%         fprintf(stderr, "ASSERT at %s:%d:\n", __FILE__, __LINE__); \
%         panic(); \
%     } 
% #define ASSERTM(expr, msg) { \
%     if (!(expr)) { \
%         fprintf(stderr, "ASSERTM: %*s\nat %s:%d:\n", (int)(sizeof(msg)-1), msg, __FILE__, __LINE__); \
%         panic(); \
%     } \
% }
% #define ASSERT_OK(expr) { \
%     auto _e_ = (expr); \
%     if (IS_ERR(_e_)) { \
%         fprintf(stderr, "ASSERT_OK at %s:%d:\n", __FILE__, __LINE__); \
%         panic(); \
%     } \
% }
% \end{minted} 

% \verb|IS_OK|, \verb|IS_ERR| - проверяют была ли ошибка

% \verb|TRY| - если была ошибка верни ошибку из текущей функции 

% \verb|ASSERT_OK| - если была ошибка, критическое завершение программы 

% \subsubsection{Контекст}
% Используется глобальный контекст, уникальный для каждого потока:

% \beginminteddef{c}
% __thread struct {
%     Allocator global_alloc;

%     Arena imm_str_arena;
%     Allocator imm_str_alloc;
%     slice_T(u8_t) dump_buffer;

%     void (*raise)(Error);

%     StreamWriter stdout_sw;
%     StreamWriter stderr_sw;
% } g_ctx;
% \end{minted} 

% \subsubsection{Аллокатор}

Вдохновленно походом, предложенным в языке Zig, каждая функция выделяющая память принимает на вход абстрактный аллокатор,
что одновременно указывает на то, что функция может выделять память(является своеобразной разметкой), 
а также является гибким решением при работе с паматью.

% Используется интерфейс абстрактного аллокатора:

% \beginminteddef{c}
% AllocatorError
% allocator_alloc(Allocator* self, usize_t size, usize_t alignment, void **out_ptr);
% AllocatorError
% allocator_resize(Allocator* self, usize_t size, usize_t alignment, void **in_out_ptr);
% allocator_free(Allocator* self, void **ptr);

% AllocatorError
% allocator_alloc_z(Allocator* self, usize_t size, usize_t alignment, void **out_ptr);
% \end{minted} 

% Примерами такого аллокатора служат глобальный glibc аллокатор и арена алокатор 

% \subsubsection{Арена Памяти}
% Для оптимизации и удобства работы с памятью используется структура арены памяти.

% \beginminteddef{c}
% struct_def(ArenaChunk, {
%     u8_t *cursor;
%     usize_t data_size;
%     u8_t data[];
% })

% struct_def(Arena, {
%     list_T(ArenaChunk) chunks;
%     usize_t default_chunk_data_size;
% })
% \end{minted} 


% \subsubsection{Строки}
% Строки и строковые "срезы" (позразумевается, что \textit{String} владеет паматью, т.к. хранит абстрактный аллокатор, 
%  а \textit{str\_t} нет)

% \beginminteddef{c}
% struct_def(String, {
%     uchar_t *ptr;
%     usize_t byte_cap;
%     usize_t byte_len; // in bytes
%     Allocator allocator;
% })

% struct_def(str_t, {
%     uchar_t *ptr;
%     usize_t byte_len; // in bytes
% })

% typedef uint32_t rune_t;
% \end{minted}

% Подразумевается, что строки используют UTF-8 кодировку.

% Тип \verb|rune_t| - unicode code point, число кодирующее символ в стандарте Юникод.

% Интерфейс итераторов считывания/записи юникод символов:
% \beginminteddef{c}
% UTF8_Error
% str_next_rune(str_t self, rune_t *out_rune, str_t *out_self);

% /// @param[in, out] out_str should be preinit, len 4 guaranties success
% UTF8_Error
% str_encode_next_rune(str_t self, rune_t rune, str_t *out_self);
% \end{minted}


% \subsubsection{Потоки ввода вывода}

% Интерфейс абстрактного потока вывода:

% \beginminteddef{c}
% enum_def(IOError, 
%     IO_ERROR_OK,
%     IO_ERROR_WRITE,
% )
% #define IO_ERROR(ERR) ((IOError)IO_ERROR_##ERR)

% typedef IOError (StreamWriter_WriteFn)(void *, usize_t, uint8_t[]);
% typedef IOError (StreamWriter_FlushFn)(void *);

% struct_def(StreamWriter_VTable, {
%     StreamWriter_WriteFn *write;
%     StreamWriter_FlushFn *flush;
% })
% struct_def(StreamWriter, {
%     StreamWriter_VTable _vtable;

%     void *data;
% })
% \end{minted}

% Тип \verb|OutputFileStream| - поток вывода в файл, реализующий привиденный ранее интерфейс.

% \beginminteddef{c}
% struct_def(OutputFileStream, {
%     slice_T(u8_t) buffer;
%     void *b_cursor;
%     void *e_cursor;
%     FILE *file;
% })

% AllocatorError
% output_file_stream_new_in(FILE *ofile, usize_t buffer_size, Allocator alloc[non_null], OutputFileStream *out_self);
% IOError
% output_file_stream_write(OutputFileStream self[non_null], usize_t data_size, u8_t data[data_size]);
% IOError
% output_file_stream_flush(OutputFileStream self[non_null]);
% StreamWriter
% output_file_stream_stream_writer(OutputFileStream self[non_null]);
% \end{minted}

% Строки тоже реализуют этот интерфейс(приведено с имплементацией).

% \beginminteddef{c}
% IOError
% output_string_stream_write(String self[non_null], usize_t data_size, u8_t data[data_size]);
% IOError
% output_string_stream_flush(String self[non_null]);
% StreamWriter
% string_stream_writer(String self[non_null]);

% IOError
% output_string_stream_write(String self[non_null], usize_t data_size, u8_t data[data_size]) {
%     ASSERT_OK(string_reserve_cap(self, data_size));
%     ASSERT_OK(string_append_str(self, (str_t) {.ptr = data, .byte_len = data_size}));
%     return IO_ERROR(OK);
% }
% IOError
% output_string_stream_flush(String self[non_null]) {
%     return IO_ERROR(OK);
% }
% StreamWriter
% string_stream_writer(String self[non_null]) {
%     return (StreamWriter) {
%         ._vtable = (StreamWriter_VTable) {
%             .write = (StreamWriter_WriteFn *)output_string_stream_write,
%             .flush = (StreamWriter_FlushFn *)output_string_stream_flush,
%         },
%         .data = (void *)self,
%     };
% }
% \end{minted}


% Используются следующие коды, для изменения цвета текста, при вывод в терминал.
% \beginminteddef{c}
% #define KNRM  "\x1B[0m"
% #define KRED  "\x1B[31m"
% #define KGRN  "\x1B[32m"
% #define KYEL  "\x1B[33m"
% #define KBLU  "\x1B[34m"
% #define KMAG  "\x1B[35m"
% #define KCYN  "\x1B[36m"
% #define KWHT  "\x1B[37m"
% \end{minted}

% \subsubsection{Форматирование строк}

% Для форматирования строк используется тип \verb|StringFormatter|, содержащий текущее состояние форматирования и настройки форматирования:
% строку для индентации(например \verb|"    "|), выходной поток, куда форматтер выводит результат.

% \beginminteddef{c}
% enum_def(FmtError,
%     FMT_ERROR_OK,
%     FMT_ERROR_ERROR,
% )

% struct_def(StringFormatter, {
%     usize_t pad_level;
%     str_t pad_string;

%     StreamWriter target;

%     bool is_line_padded;
% })
% \end{minted}

% Основыные методы типа \verb|StringFormatter|:
% \beginminteddef{c}
% FmtError
% string_formatter_write(StringFormatter *fmt, const str_t s);
% FmtError
% string_formatter_writeln(StringFormatter *fmt, const str_t s);
% FmtError
% string_formatter_write_fmt(StringFormatter *fmt, str_t fmt_str, ...);
% \end{minted}

% Выше \verb|write_fmt| реализует функционал схожий с функций \verb|glibc| \verb|printf|, 
% добавлена поддержка строк типа \verb|str_t| и объектов реализующих интерфейс \verb|Formattable|.

% Интерфейс \verb|Formattable|:
% \beginminteddef{c}
% FmtError 
% formattable_fmt(Formattable *self, StringFormatter *fmt);
% \end{minted}


% Макросы для вывода:
% \beginminteddef{c}
% #define dbgp(___prefix, val, args...) { \
%     struct dbgp_args\
%     {\
%         typeof(val) _val;\
%         void *data;\
%     };\
%     auto _args = ((struct dbgp_args) { val, args});\
%     auto fmt = string_formatter_default(&g_ctx.stdout_sw); \
%     ASSERT_OK(___prefix##_dbg_fmt(_args._val, &fmt, _args.data));  \
%     ASSERT_OK(string_formatter_write(&fmt, S("\n")));      \
%     ASSERT_OK(stream_writer_flush(&fmt.target)); \
% } \

% #define print_pref(___prefix, val) { \
%     auto fmt = string_formatter_default(&g_ctx.stdout_sw); \
%     ASSERT_OK(___prefix##_fmt((val), &fmt, nullptr)); \
%     ASSERT_OK(stream_writer_flush(&fmt.target)); \
% } \

% #define println_pref(___prefix, val) { \
%     auto fmt = string_formatter_default(&g_ctx.stdout_sw); \
%     ASSERT_OK(___prefix##_fmt((val), &fmt, nullptr)); \
%     ASSERT_OK(string_formatter_write(&fmt, S("\n"))); \
%     ASSERT_OK(stream_writer_flush(&fmt.target)); \
% }
% \end{minted}

% Макросы для форматированного вывода:
% \beginminteddef{c}
% #define print_fmt(fmt_str, args...) { \
%     auto fmt = string_formatter_default(&g_ctx.stdout_sw); \
%     ASSERT_OK(string_formatter_write_fmt(&fmt, fmt_str, ##args)); \
%     ASSERT_OK(stream_writer_flush(&fmt.target)); \
% }                                                                     
% #define println_fmt(fmt_str, args...) { \
%     auto fmt = string_formatter_default(&g_ctx.stdout_sw); \
%     ASSERT_OK(string_formatter_write_fmt(&fmt, fmt_str, ##args)); \
%     ASSERT_OK(string_formatter_write(&fmt, S("\n"))); \
%     ASSERT_OK(stream_writer_flush(&fmt.target)); \
% }                                                                     

% #define eprint_fmt(fmt_str, args...) { \
%     auto fmt = string_formatter_default(&g_ctx.stderr_sw); \
%     ASSERT_OK(string_formatter_write_fmt(&fmt, fmt_str, ##args)); \
%     ASSERT_OK(stream_writer_flush(&fmt.target)); \
% }                                                                     
% #define eprintln_fmt(fmt_str, args...) { \
%     auto fmt = string_formatter_default(&g_ctx.stderr_sw); \
%     ASSERT_OK(string_formatter_write_fmt(&fmt, fmt_str, ##args)); \
%     ASSERT_OK(string_formatter_write(&fmt, S("\n"))); \
%     ASSERT_OK(stream_writer_flush(&fmt.target)); \
% }                                                                     
% \end{minted}

% Реализацию можно посмотреть в приложении[\ref{extras:fmt_impl}]

% \subsubsection{Динамическая таблица символов}

% Динамическая(runtime) таблица символов используется для параметризации некоторых контейнеров по типу.

% \beginminteddef{c}
% struct_def(TypeInfo_VTable, {
%     FmtFn *fmt;
%     FmtFn *dbg_fmt;

%     EqFn *eq;
%     SetFn *set;
%     HashFn *hash;
% })
% struct_def(TypeInfo, {
%     usize_t size;
%     usize_t align;
%     str_t name;

%     TypeInfo_VTable _vtable;
% })

% #define typeid_of(T) TYPE_ID_##T

% #ifndef TYPE_LIST
% #define TYPE_LIST \
%     TYPE_LIST_ENTRY(int), \
%     TYPE_LIST_ENTRY(usize_t), \
%     TYPE_LIST_ENTRY(str_t), \
%     TYPE_LIST_ENTRY(ArenaChunk), \
%     TYPE_LIST_ENTRY(darr_t)

% #endif // TYPE_LIST

% typedef enum TypeId TypeId;
% enum TypeId {
% #define TYPE_LIST_ENTRY(T) typeid_of(T)
%     TYPE_LIST
% #undef TYPE_LIST_ENTRY
% };
% \end{minted}

% \subsubsection{Статический и динамический массивы}

% Тип статического массива или "среза памяти":

% \beginminteddef{c}
% struct_def(SliceVES, { 
%     void *ptr;             
%     usize_t len;        

%     TypeId el_tid;
%     usize_t el_size;
%     usize_t el_align;
% })

% typedef SliceVES slice_t;
% #define slice_T(T) slice_t
% \end{minted}

% Подразумевается, что срез не владеет памятью.

% Тип динамического массива:

% \beginminteddef{c}
% struct_def(DArrVES, {        
%     slice_t data;
%     usize_t len;        
%     Allocator allocator;
% })
% \end{minted}

% Принимает произвольный аллокатор как параметр.

% \subsubsection{Хеш-таблицы}
% \label{prim:hashmap}

% Реализацию приведенна в проекте[\ref{extras:ecc}]

% \beginminteddef{c}
% struct_def(HashMap, {
%     SliceVES_T(HashMap_Bucket) buckets;
%     SliceVES keys;
%     SliceVES values;

%     usize_t count;
    
%     HashFn *key_hash;
%     EqFn *key_eq;
%     SetFn *key_set;

%     SetFn *value_set;
%     TypeId key_tid;
%     TypeId value_tid;

%     Allocator alloc;
% })

% typedef HashMap * hashmap_t;

% // hashmap_T(str_t, any_t)
% #define hashmap_T(key, val) hashmap_t
% \end{minted}

% \newpage
% \subsection{Интерфейс работы с узлом трансляции}

% В библиотеке есть высокоуровневая абстракция единицы трансляции, на которой реализованны функции преобразований(прохов/passes).


% \beginminteddef{c}
% struct_def(C_TranslationUnitData, {
%     str_t main_file;
%     hashmap_T(str_t, TU_FileData) file_data_table;
%     // lexing
%     darr_T(C_Token) tokens;
%     // parsing
%     C_Ast_TranslationUnit *tr_unit;

%     // analysis
%     C_SymbolTable symbol_table;
%     // hashmap_T(str_t, void) topsort_start_symbols;
% #ifdef EXTENDED_C
%     C_SymbolTable proc_macro_table;
%     void *proc_macro_lib; // handle from dlopen call
% #endif // EXTENDED_C

%     // mem
%     Arena string_arena;
%     Arena token_arena;
%     Arena ast_arena;
% })

% void
% c_translation_unit_init(C_TranslationUnitData *self, str_t main_file_path);
% void
% c_translation_unit_deinit(C_TranslationUnitData *self);
% bool
% c_translation_unit_lex(C_TranslationUnitData *self);
% bool
% c_translation_unit_parse(C_TranslationUnitData *self);
% void
% ec_translation_unit_ast_unparse(C_TranslationUnitData *self, StreamWriter *dst_sw);
% void
% ec_translation_unit_ast_compile_graphvis(C_TranslationUnitData *self, StreamWriter *dst_sw);
% \end{minted}

