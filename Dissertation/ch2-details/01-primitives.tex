
\clearpage
\section{Основные примитивы}
\label{primitives}

Библиотека реализованна на языке C23, используется некоторые улучшения, добавленные в новом стандарте.

Используются:
\begin{enumerate}
  \item ключевое слово \textbf{auto} для автоматического вывода типов
  \item взятие адреса у литерала, например \verb|&(int) {3}|
  \item gcc выражения-утверждения (statement expressions) вида \verb|({x = 3; x;})|
  \item gcc \verb|,##| оператор в макросах
\end{enumerate}

Для написания основной библиотеки используются дочерняя библиотека[\ref{extras:c-core}] с основными примитивами: 
строковыми, ввода-вывода, аллокации памяти и др.

Данная библиотека реализована в виде единичных самодостаточных заголовочных файлов, 
которые могут быть в зависимости от флага препроцессора либо заголовочными файлами либо файлами имплементации, 
по принципу stb библиотек\cite{stb_libs}.

Далее рассматриваю каждый примитив в отдельности:
\begin{itemize}
\item Макросы определения составных типов

Для определений структур и энумераций используются следующие макросы препроцессора Си[\ref{primitives:def-macros}]:
\begin{lstlisting}[language=c, caption={Макросы определения составных типов}, label={primitives:def-macros}]
#define struct_decl(name) \
typedef struct name name; \
struct name; \

#define enum_decl(name) \
typedef enum name name; \
enum name; \

#define struct_def(name, fields) \
typedef struct name name; \
struct name fields; \

#define enum_def(name, ...) \
typedef enum name name; \
enum name {__VA_ARGS__}; \
\end{lstlisting} 

Данные макросы позволяют не использовать соответствующее ключевое слово перед именем типа, 
например \verb|A| вместо \verb|struct A|.

\item Обработка ошибок


Ошибки кодируются как enum, например:

\begin{lstlisting}[language=c, caption={Ошибки Аллокатора}, label={primitives:error-enum-ex}]
enum_def(AllocatorError,
    ALLOCATOR_ERROR_OK,
    ALLOCATOR_ERROR_MEM_ALLOC,
    ALLOCATOR_ERROR_COUNT
)
#define ALLOCATOR_ERROR(ERR) ((AllocatorError)ALLOCATOR_ERROR_##ERR)
\end{lstlisting}

Значение OK кодируется как 0, остальные значения кодируются положительными числами.

Для обработки ошибок используются следующие макросы:
% \begin{minted}[linenos, frame=single]{c}
\begin{lstlisting}[language=c, caption={Макросы обработки ошибок}, label={primitives:error-macros}]
#define IS_OK(expr) ...
#define IS_ERR(expr) ...
#define TRY(expr) ...
#define ASSERT(expr) ...
#define ASSERTM(expr, msg) ...
#define ASSERT_OK(expr) ...
#define DBG_ASSERT(expr) ...
\end{lstlisting}

Где
\begin{itemize}
    \item \verb|IS_OK| - возвращает \verb|true|, если не было ошибки в данном выражении.

    \item \verb|IS_ERR| - возвращает \verb|true|, если была ошибка в данном выражении.

    \item \verb|TRY| - если в данном выражение была ошибка, то возвращает ошибку из текущей функции ключевым словом \verb|return|.

    \item \verb|ASSERT| - если выражение ложно, то критически завершает программу, используя вызов функции \verb|panic|.
    Данный примитив используется для проверки инвариантов во время работы программы и не убирается в \verb|release| режиме сборки.

    \item \verb|ASSERTM| - ведет себя также как \verb|ASSERT|, но позволяет дополнительно передать сообщение об ошибке.

    \item \verb|ASSERT_OK| - идентичен композиции \verb|ASSERT(IS_OK(expr))|.
    Является аналогом метода \verb|unwrap| в языке Rust.

    \item \verb|DBG_ASSERT| - аналог \verb|ASSERT|, убирающийся в \verb|release| режиме сборки.
\end{itemize}

\item Контекст

Используется глобальный контекст, уникальный для каждого потока:

\begin{lstlisting}[language=c, caption={Структура глобального контекста}, label={primitives:g_ctx-struct}]
__thread struct {
    Allocator global_alloc;

    StreamWriter stdout_sw;
    StreamWriter stderr_sw;
} g_ctx;
\end{lstlisting} 

Данный контекст содержит глобальный аллокатор[\ref{primitives:allocator}], и потоки вывода для \verb|stdout|, \verb|stderr|.

\item\label{primitives:allocator} Аллокатор

Используется интерфейс абстрактного аллокатора:

\begin{lstlisting}[language=c, caption={Макросы обработки ошибок}, label={primitives:error-macros}]
AllocatorError
allocator_alloc(Allocator* self, usize_t size, usize_t alignment, void **out_ptr);
AllocatorError
allocator_resize(Allocator* self, usize_t size, usize_t alignment, void **in_out_ptr);
allocator_free(Allocator* self, void **ptr);

AllocatorError
allocator_alloc_z(Allocator* self, usize_t size, usize_t alignment, void **out_ptr);
\end{lstlisting} 

Где данные функции работают аналогично функциям \verb|glibc| \verb|malloc|, \verb|resize|, \verb|free|, \verb|calloc| соответственно.
Однако они также поддерживаю произвольное выравнивание типа.

Примерами такого аллокатора служат глобальный glibc аллокатор и арена аллокатор[\ref{primitives:arena}].

\item\label{primitives:arena} Арена Памяти

Для оптимизации и удобства работы с памятью используется тип арены памяти \verb|Arena|, реализующая интерфейс абстрактного аллокатора.
Арена заранее выделяет большой кусок(chunk) памяти и при запросе на выделение памяти(функцией \verb|arena_alloc|) дает указатель внутри него. Если текущего куска недостаточно выделяется еще один и т.д.
Все куски хранятся как связанный список.

При запросе на освобождение(функцией \verb|arena_free|) памяти ничего не происходит. Арена очищает всю память разом при вызове метода \verb|arena_reset|.

Такой интерфейс позволяет удобно выделять память под задачи, имеющие определенное, конечное время жизни, как например этапы компиляции.

\item Строки

Привожу определения строк и строковых "срезов":

\begin{lstlisting}[language=c, caption={Строки и строковые срезы}, label={primitives:string-struct}]
struct_def(String, {
    uchar_t *ptr;
    usize_t byte_cap;
    usize_t byte_len; // in bytes
    Allocator allocator;
})

struct_def(str_t, {
    uchar_t *ptr;
    usize_t byte_len; // in bytes
})

typedef uint32_t rune_t;
\end{lstlisting}

Подразумевается, что \verb|String| владеет памятью, т.к. хранит абстрактный аллокатор, а \verb|str_t| нет.
Тип \verb|String| ведет себя в точности как динамический массив символов.

Подразумевается, что строки используют UTF-8 кодировку.

Тип \verb|rune_t| - unicode code point, число кодирующее символ в стандарте Юникод.

Строки - основной примитив использованный в данной работе. Они являются альтернативой Си строкам, которые не имеют поля длины и являются ноль-терминированными.
Для конвертации строковых литералов Си к типу \verb|str_t| используется макрос препроцессора Си: \verb|S(str)|.

При лексическом разборе[\ref{pass:lexing}] используется интерфейс итераторов считывания/записи юникод символов:
\begin{lstlisting}[language=c, caption={Функиции итерации считывания/записи строк}, label={primitives:str-iter-api}]
UTF8_Error
str_next_rune(str_t self, rune_t *out_rune, str_t *out_self);
UTF8_Error
str_encode_next_rune(str_t self, rune_t rune, str_t *out_self);
\end{lstlisting}


\item Потоки вывода

Далее привожу интерфейс абстрактного потока вывода, использующийся для:

\begin{lstlisting}[language=c, caption={Интерфейс абстрактного потока вывода}, label={primitives:io-api}]
typedef IOError (StreamWriter_WriteFn)(void *, usize_t, uint8_t[]);
typedef IOError (StreamWriter_FlushFn)(void *);
\end{lstlisting}


Тип \verb|OutputFileStream| - поток вывода в файл, реализующий приведенный ранее интерфейс.
Строки тоже реализуют этот интерфейс(приведено с имплементацией).

\begin{lstlisting}[language=c, caption={Реализация интерфейса потока вывода другими типами}]
StreamWriter
output_file_stream_stream_writer(OutputFileStream *self);
StreamWriter
string_stream_writer(String *self);
\end{lstlisting}

\item Форматирование строк

Для форматирования строк используется тип \verb|StringFormatter|, содержащий текущее состояние форматирования и настройки форматирования:
строку для индентации(например \verb|"    "|), выходной поток, куда форматтер выводит результат.

Основыные методы типа \verb|StringFormatter|:
\begin{lstlisting}[language=c, caption={Интерфейс объекта форматирования строк}, label={primitives:string-formatter-api}]
FmtError
string_formatter_write(StringFormatter *fmt, const str_t s);
FmtError
string_formatter_writeln(StringFormatter *fmt, const str_t s);
FmtError
string_formatter_write_fmt(StringFormatter *fmt, str_t fmt_str, ...);
\end{lstlisting}

Выше \verb|write_fmt| реализует функционал схожий с функцией \verb|glibc| \verb|printf|, 
добавлена поддержка строк типа \verb|str_t| и объектов реализующих интерфейс \verb|Formattable|[\ref{primitives:formattable-api}].

Интерфейс абстрактного форматируемого объекта \verb|Formattable|:
\begin{lstlisting}[language=c, caption={Интерфейс абстрактного форматируемого объекта}, label={primitives:formattables-api}]
FmtError 
formattable_fmt(Formattable *self, StringFormatter *fmt);
\end{lstlisting}


Используются следующие макросы для вывода в консоль:
\begin{lstlisting}[language=c, caption={Макросы для вывода в консоль}, label={primitives:print-macros}]
#define dbgp(___prefix, val, args...) ...
#define print_pref(___prefix, val) ...
#define println_pref(___prefix, val) ...

#define print_fmt(fmt_str, args...) ...                                 
#define println_fmt(fmt_str, args...) ...                               

#define eprint_fmt(fmt_str, args...) ...                                
#define eprintln_fmt(fmt_str, args...) ...                              
\end{lstlisting}

Где
\begin{itemize}
    \item \verb|dbgp| - макрос отладочного вывода. Вызов вида \verb|dbgp(foo, val, args...)| вызовет функцию отладочного форматирования \verb|foo_dbg_fmt|(ожидается, что она существует).
    \item \verb|print_pref| - макрос вывода в стандартный поток вывода по префиксу функции. Вызов вида \verb|print(foo, val)| вызовет функцию форматирования объекта \verb|foo_fmt|(ожидается, что она существует).
    \item \verb|println_pref| - аналогичен \verb|print_pref|, но пишет символ переноса строки в конце.

    \item \verb|print_fmt| - работает аналогично функции \verb|glibc| \verb|printf|, но вместо строк Си использует строки типа \verb|str_t|, также поддерживает абстрактные форматируемые объекты.
    \item \verb|println_fmt| - аналогичен \verb|print_fmt|, но пишет символ переноса строки в конце.
    
    \item \verb|eprint_fmt| - аналогичен \verb|print_fmt|, но пишет в стандартный поток ошибок.
    \item \verb|eprintln_fmt| - аналогичен \verb|println_fmt|, но пишет в стандартный поток ошибок.
\end{itemize}

\item Структуры данных

Далее в работе активно используются следующие структуры данных:
\begin{itemize}
    \item \verb|slice_t| - статический массив, выполнен как указатель и кол-во элементов.
    \item \verb|darr_t| - динамический массив, владеет паматью и хранит абстрактный аллокатор.
    \item \verb|hashmap_t| - хеш-таблица.
    \item \verb|hashset_t| - хеш-множество.
\end{itemize}


Для каждого из них существует аналогичный макрос с окончанием \verb|_T|, позволяющий указывать имена типов, находящихся в контейнере.
Например \verb|darr_T(int)| - динамический массив целых чисел, \verb|hashmap_T(str_t, Foo)| - хеш-таблица с ключами типа \verb|str_t| и со значениями типа \verb|Foo|.

%

% \newpage
% \subsection{Интерфейс работы с узлом трансляции}

% В библиотеке есть высокоуровневая абстракция единицы трансляции, на которой реализованны функции преобразований(прохов/passes).


% \beginminteddef{c}
% struct_def(C_TranslationUnitData, {
%     str_t main_file;
%     hashmap_T(str_t, TU_FileData) file_data_table;
%     // lexing
%     darr_T(C_Token) tokens;
%     // parsing
%     C_Ast_TranslationUnit *tr_unit;

%     // analysis
%     C_SymbolTable symbol_table;
%     // hashmap_T(str_t, void) topsort_start_symbols;
% #ifdef EXTENDED_C
%     C_SymbolTable proc_macro_table;
%     void *proc_macro_lib; // handle from dlopen call
% #endif // EXTENDED_C

%     // mem
%     Arena string_arena;
%     Arena token_arena;
%     Arena ast_arena;
% })

% void
% c_translation_unit_init(C_TranslationUnitData *self, str_t main_file_path);
% void
% c_translation_unit_deinit(C_TranslationUnitData *self);
% bool
% c_translation_unit_lex(C_TranslationUnitData *self);
% bool
% c_translation_unit_parse(C_TranslationUnitData *self);
% void
% ec_translation_unit_ast_unparse(C_TranslationUnitData *self, StreamWriter *dst_sw);
% void
% ec_translation_unit_ast_compile_graphvis(C_TranslationUnitData *self, StreamWriter *dst_sw);
% \end{minted}

% Примеры использования данного иМнтерфейса: % TODO
\end{itemize}