
% \section{Грамматический разбор}
% \label{pass:parsing}

% \subsection{Парсер Пратта}
% \label{pass:parsing:pratt}
% Парсер Пратта использует комбинацию циклов и рекурсивных вызовов функций.
% Есть хорошая статья, в которой приводится реализация данного парсера на языке Rust\cite{matklad_pratt_parsers}.

% Парсер Пратта для инфиксных операторов упрощенно работает по следующему принципу: 

% % \begin{lstlisting}[language=python, caption={Псевдокод алгоритма парсера Пратта}]
% % def parse_expr(state, max_precedence, flags) -> expr, err:
% %     left: Expr
    
% % \end{lstlisting}

% \begin{enumerate}
%     \item Начало определения рекурсивной функции \verb|parse_expr|, 
%     принимающей параметры состояния разбора \verb|state| и максимального значения порядка следования \verb|max_precedence|

%     \begin{itemize}
%         \item Пусть \verb|left: Expr| - переменная типа \verb|Expr|

%         \item В \verb|left| записывается первый аргумент или выходим с ошибкой, если его нет или он является именем типа

%         \item\label{pratt:alg:loop} Пока есть следующий за аргументом бинарный оператор, удовлетворяющий ограничениям порядка следования и ассоциативности:

%         \begin{itemize}
%             \item Разбираем оператор записываем в переменную \verb|op: Expr|

%             \item\label{pratt:alg:rec-call} Разбираем правый операнд выражения рекурсивным вызовом функции \verb|parse_expr|, 
%             со значением \verb|max_precedence| равным порядку следования оператора в переменной \verb|op|

%             \item Записываем в \verb|left| значение переменной \verb|op|
%         \end{itemize}
%     \end{itemize}

%     \item Конец определения функции \verb|parse_expr|
% \end{enumerate}


% Рекурсивные вызовы[\ref{pratt:alg:rec-call}] в алгоритме сами по себе разбирают цепочки операторов с убывающем номером порядка следования
% (или одинаковым при условии ассоциативности справа на лево). 
% Синтаксические деревья полученные после разбора таких операторов, являются наклоненными направо, как показано на диаграмме[\ref{pratt:right-lean-diag}].

% Тогда как цикл[\ref{pratt:alg:loop}] в алгоритме сам по себе разбирает цепочки операторов с оставшимися после рекурсивного вызова номерами порядка следования, 
% т.е. возрастающими или одинаковыми при условии ассоциативности слева на право. 
% Синтаксические деревья полученные после разбора таких операторов, являются наклоненными налево, как показано на диаграмме[\ref{pratt:left-lean-diag}].


% \begin{figure}[h!]
%     \includegraphics[width=\textwidth,height=\textheight,keepaspectratio]{right-lean-diag.png}
%     \centering
%     \caption{AST цепочки операторов с убывающим номером порядка следования}
%     \label{pratt:right-lean-diag}
% \end{figure}

% \begin{figure}[h!]
%     \includegraphics[width=\textwidth,height=\textheight,keepaspectratio]{left-lean-diag.png}
%     \centering
%     \caption{AST цепочки операторов с возрастающим номером порядка следования}
%     \label{pratt:left-lean-diag}
% \end{figure}


% \FloatBarrier




\clearpage
\section{Грамматический разбор}
\label{pass:parsing}

% Грамматический разбор 

% Метод рекурсивного спуска


Грамматика фразовых структур языка Си, приведенная в стандарте\cite{c99_std} (Annex A, 2. Phrase structure grammar),
по большей части является контекстно свободной, однако зависимость от контекста, можно показать на следующем примере(пример взять из статьи\cite{eli_c_cs}):

\begin{lstlisting}[language=c]
(T) * x;
\end{lstlisting}

В зависимости от того, что такое \verb|T| (идентификатор или имя типа) данное выражение может быть разобрано по разному:
как умножение двух переменных
\begin{lstlisting}[language=c]
T * x;
\end{lstlisting}
или как оператор приведения к типу (T), примененному к разыменовыванию указателя \verb|x|
\begin{lstlisting}[language=c]
(T) (*x);
\end{lstlisting}

Это проиллюстрировано в следующем изображение[\ref{parsing:gram-dep-diag}]:
\begin{figure}[h!]
    \includegraphics[width=\textwidth,height=\textheight,keepaspectratio]{gram-dep-diag.png}
    \centering
    \caption{Зависимость грамматики выражений Си от контекста}
    \label{parsing:gram-dep-diag}
\end{figure}
\FloatBarrier


Далее в Си есть понятие затенения(shadowing) имен. 
Следующий пример демонстрирует, что имя типа может быть затенено (использовался компилятор gcc):

\begin{lstlisting}[language=c, caption={Пример затенения имен}, label={parsing:name-shadowing-ex}]
typedef int Foo2;

void test_type_name() {
    int Foo2 = 3;
    int x;
    x = (Foo2) + x;
}
\end{lstlisting}

\verb|Foo2| внутри функции \verb|test_type_name| затеняет \verb|typedef| определение, и выражение \verb|(Foo2) + x| разбирается как сумма, 
а не приведение к типу \verb|Foo| и унарный плюс.

Так что для корректного разбора выражений языка Си, нужна система разрешения имен.

Данная система выполнена в виде стека таблиц символов(далее тип \verb|C_Environment|).
Внизу стека лежит глобальная таблица символов, где хранятся определения типов и функций. Далее по стеку идут локальные таблицы символов, соответствующие вложенным блокам кода(scopes).

\begin{lstlisting}[language=c, caption={Определения структур сивола и окружения}, label={parsing:env-struct}]
typedef str_t C_Symbol;
typedef hashmap_T(C_Symbol, C_SymbolData) C_SymbolTable;

struct_def(C_SymbolData, {
    C_Ast_Node *node;
    ...
})

typedef darr_T(C_SymbolTable *) C_Environment;
\end{lstlisting}

Для разрешения имени функция \verb|c_environment_get_sym_data| идет по стеку сверху вниз, проверяя принадлежит ли имя текущему блоку.
Так при определении нового имени в текущем блоке, оно перекроет имена находящиеся в нижележащих блоках. 

\begin{lstlisting}[language=c, caption={Определение функции разрешения имени символа}, label={parsing:get-sym-data-def}]
C_SymbolData NLB(*)
c_environment_get_sym_data(C_Environment env, C_Symbol sym) {
    C_SymbolData *data = nullptr;
    for (isize_t i = (isize_t)darr_len(env) - 1; i >= 0; i -= 1) {
        data = hashmap_get(**darr_get_T(C_SymbolTable *, env, i), &sym);
        if (data != nullptr) {
            break;
        }
    }
    return data;
}
\end{lstlisting}


% \subsection{Структура парсера}

Также как на этапе лексического разбора[\ref{pass:lexing}] на данном этапе имеется структура содержащая состояние разбора.

\begin{lstlisting}[language=c]
struct_def(ParserState, { ... })
\end{lstlisting}

Данная структура содержит информацию о:
\begin{itemize}
    \item последовательности разбираемых токенов
    \item номер текущего разбираемого токена
    \item текущее окружение, для разрешения имен
    \item флаг, указывающий, откладывает ли парсер разбор контекстно зависимых частей языка
    \item флаг, указывающий, находится ли парсер в режиме ошибки
    \item аллокаторы для строк и элементов AST
    \item функции обработки ошибок аллокации и возникающих при разборе
\end{itemize}



% \subsubsection{Обработка ошибок}

% Когда какая-нибудь функция разбора возвращается ошибку, она дополнительно может привести сообщение об ошибке с помощью макросов:

% \beginminteddef{c}
% #define parser_error(state, _msg, args...) { \
%     if (!state->was_error) { \
%         (state)->last_error = (ParsingErrorData) { \
%             .pos = parser_pos(state), \
%             .msg = (_msg), \
%             .msg_kind = PARSING_ERROR_MESSAGE_KIND_NORMAL, \
%             ##args \
%         }; \
%         (state)->was_error = true; \
%     } \
% }
% #define parser_error_pos(state, _pos, _msg, args...) ...
% #define parser_error_expected(state, expected, args...) ... 
% \end{minted}

% При выполнении эти макросы ставят флаг \verb|ParserState::was_error|, сообщая дальнейшим процессам о наличии ошибки.
% При этом ошибка устанавливается, только если нет активной ошибки (флаг \verb|ParserState::was_error| равен \verb|false|), так в цепочке ошибок до конца дойдет только первая встретившаяся, 
% находящаяся на глубине стека вызовов и соответственно самая частная, зачастую несущая информацию о том, что пользователю надо изменить в программе, чтобы та была корректной.

% В добавок к этому функции разбора возвращают статут разбора при завершении:
% \beginminteddef{c}
% enum_def(ParsingError, 
%     PARSING_ERROR_OK,
%     PARSING_ERROR_NONE,
%     PARSING_ERROR_EOF,
% )
% #define PARSING_ERROR(ERR) ((ParsingError)PARSING_ERROR_##ERR)
% \end{minted}

% Для обработки вышеперечисленного вызывающими функциями(callers) используются следующие макросы: 
% \beginminteddef{c}
% #define PARSER_ALLOC_HANDLE(f) ...
% #define PARSER_TRY(p) ... 
% #define PARSER_OPTIONAL(p) ...
% #define PARSING_OK(state) ...
% #define PARSING_NONE(state, prev) ...
% \end{minted}

% Где
% \begin{itemize}
% \item \verb|PARSER_ALLOC_HANDLE| - 
% \item \verb|PARSER_TRY| - при неудачном завершении вызываемого, отдаст ошибку далее по стеку
% \item \verb|PARSER_OPTIONAL| - при неудачном завершение вызываемого, "заглушит" ошибку
% \item \verb|PARSER_OK| - выйдет из функции с "успехом" 
% \item \verb|PARSER_NONE| - выйдет из функции с "неудачей", восстановив состояние парсера до начала разбора текущей функцией
% \end{itemize}

% \beginminteddef{c}
% \end{minted}

\subsection{Абстрактное синтаксическое дерево}

Грамматика языка Си приведена в CFG форме, каждый нетерминал грамматики по сути является элементом синтаксического дерева, 
однако после разбора такое дерево сдержит множество излишних деталей, подробностей разбора.
Разработанное абстрактное синтаксическое дерево(AST) является "выжимкой" из синтаксического дерева и будет рассмотрено далее.

Абстрактное синтаксическое дерево, получаемое в ходе грамматического разбора, состоит из элементов(nodes).
Как и в случае с лексером тип \verb|C_Ast_Node| тоже является типом объединения частных типов, которые зачастую в свою очередь тоже являются объединениями дочерних типов.
Данная иерархия будет рассмотрена далее, полный код синтаксического дерева можно посмотреть в приложении[\ref{extras:c_ast}].

Самый общий тип иерархии тип \verb|C_Ast_Node|, далее приведено определение его структуры:


\begin{lstlisting}[language=c, caption={Структура элемента AST}, label={parsing:ast:node-struct}]
#define C_AST_NODE_BASE \
    C_Ast_NodeKind kind;\
    C_ParserSpan span;\

struct_def(C_Ast_Node, {
    union {
    struct {
        C_AST_NODE_BASE
    };

        C_Ast_Ident ident;
        C_Ast_Literal lit;
        C_Ast_Expr expr;
        C_Ast_Stmt stmt;
        C_Ast_Type ty;
        C_Ast_Decl decl;
        C_Ast_TranslationUnit tr_unit;
    #ifdef EXTENDED_C
        C_Ast_AtDirective at_directive;
    #endif // EXTENDED_C
    };
})
\end{lstlisting}

Элементами его являются:
\begin{itemize}
    \item идентификатор
    \item литерал
    \item выражение
    \item утверждения
    \item имя-типа
    \item определение
    \item единица-трансляции
    \item @ директива
\end{itemize}

При разборе ноды дерева сохраняются в отдельную арену памяти(далее \verb|ParserState::ast_arena|).

Замечание: \verb|C_AST_Node| - тип переменной длины, для конкретной ноды ее размер равен размеру частной ноды,
т.е. чтобы скопировать или записать существующую ноду надо сначала понять, какого она типа.

Для построения нод используются вспомогательные макросы:
\begin{lstlisting}[language=c]
#define make_node(state, out_node, KIND_SUFF, args...) {\
    PARSER_ALLOC_HANDLE(allocator_alloc_T(&state->ast_alloc, typeof(**out_node), out_node));\
    **out_node = ((typeof(**out_node)) {.kind = C_AST_NODE_KIND_##KIND_SUFF, ##args });\
}
#define make_node_type(state, out_node, TYPE_KIND_SUFF, args...) ...
#define make_node_decl(state, out_node, DECL_KIND_SUFF, args...) ...
#define make_node_stmt(state, out_node, STMT_KIND_SUFF, args...) ...
#define make_node_expr(state, out_node, EXPR_KIND_SUFF, args...) ...
#define make_node_lit(state, out_node, LIT_KIND_SUFF, args...) ...
\end{lstlisting}

Данные макросы выделяют достаточное кол-во памяти под данный тип элемента AST, и заполняют его приведенными полями.


% \subsection{Идентификаторы и литералы}

% \beginminteddef{c}
% struct_def(C_Ast_Ident, {
%     C_AST_NODE_BASE

%     str_t name;
% })

% enum_def(C_Ast_LiteralKind, 
%     C_AST_LITERAL_KIND_INVALID,
%     C_AST_LITERAL_KIND_STRING,
%     C_AST_LITERAL_KIND_CHAR,
%     C_AST_LITERAL_KIND_NUMBER,
%     C_AST_LITERAL_KIND_COMPOUND,
% ) 
% \end{minted}

% \begin{itemize}
%     \item STRING, NUMBER, CHAR совпадают с соответствующими токенами
%     \item COMPOUND - составной литерал, например \verb|(str_t) {.ptr = nullptr, .byte_len = 0}|
% \end{itemize}

\clearpage
\subsection{Выражения}

Следующая диаграмма иллюстрирует структуру AST выражения: 

% \begin{tikzpicture}[
% roundnode/.style={ellipse, draw=black!60, fill=green!0, very thick, minimum size=7mm},
% squarednode/.style={rectangle, draw=red!60, fill=red!5, very thick, minimum size=5mm},
% ]
% %Nodes
% \node[roundnode]      (maintopic)                              {Expr};
% \node[roundnode]        (unop)       [below=of maintopic] {UnOp};
% \node[roundnode]      (binop)       [below=of maintopic] {BinOP};
% \node[roundnode]        (lowercircle)       [below=of maintopic] {4};

% %Lines
% \draw[->] (maintopic.north) -- (binop.south);
% \draw[->] (maintopic.north) -- (unop.south);
% % \draw[->] (uppercircle.south) -- (maintopic.north);
% % \draw[->] (rightsquare.south) .. controls +(down:7mm) and +(right:7mm) .. (lowercircle.east);
% \end{tikzpicture}


\begin{figure}[h!]
    \includegraphics[width=\textwidth]{expr-ast.png}
    \centering
    \caption{AST выражения Си}
    \label{parsing:expr-ast-diag}
\end{figure}

% \begin{lstlisting}[language=c]
% enum_def(C_Ast_ExprKind, 
%     C_AST_EXPR_KIND_INVALID,
%     C_AST_EXPR_KIND_IDENT,
%     C_AST_EXPR_KIND_LITERAL,
%     C_AST_EXPR_KIND_UNOP,
%     C_AST_EXPR_KIND_BINOP,
%     C_AST_EXPR_KIND_CONDOP, // ?:
%     C_AST_EXPR_KIND_CAST,
%     C_AST_EXPR_KIND_FN_CALL,
%     C_AST_EXPR_KIND_ARRAY_SUB,
%     C_AST_EXPR_KIND_COMPOUND, // expr, expr, ... expr
% )
% \end{lstlisting}

% \begin{itemize}
%   \item \verb|IDENT| - \verb|x|
%   \item \verb|LITERAL| - \verb|3|
%   \item \verb|UNOP| - \verb|-x|
%   \item \verb|BINOP| - \verb|x + y|
%   \item \verb|CONDOP| - \verb|x > y ? a : b|
%   \item \verb|CAST| - \verb|(T) expr|
%   \item \verb|FN_CALL| - \verb|f(e1, e2, e3)|
%   \item \verb|ARRAY_SUB| - \verb|arr[x]|
%   \item \verb|COMPOUND| - \verb|e1, e2, e3|
% \end{itemize}

% % \beginminteddef{c}
% % \end{minted}

% \subsubsection{Разбор выражений}

Для разбора выражений используется парсер Пратта, который хорошо сочетается с разбором методом рекурсивного спуска.
Данный парсер использует таблицу приоритета и ассоциативности операторов. Данная таблица приведена в определениях[\ref{extras:c_defs}].

Операторы разделены на три группы: префиксные, инфиксные, постфиксные.
Отдельно обрабатывается тернарный условный оператор \verb|?:|.

Для каждого из этих групп есть функция, распознающая оператор по типу с ограничением по приоритету.
Замечание: числовое значение приоритета идет(возрастает) в порядке убывания приоритета[\cite{cppref_op_prec}].

\begin{lstlisting}[language=c]
ParsingError
c_parse_expr_op_infix_prec_rb(ParserState *state, C_Ast_ExprBinOp **out_binop, u8_t max_precedence, bool is_strict_precedence);
ParsingError
c_parse_expr_op_prefix_prec_rb(ParserState *state, C_Ast_ExprUnOp **out_unop, u8_t max_precedence, bool is_strict_precedence);
ParsingError
c_parse_expr_op_postfix_prec_rb(ParserState *state, C_Ast_ExprUnOp **out_unop, u8_t max_precedence);
ParsingError
c_parse_expr_op_tern_cond_prec_rb(ParserState *state, C_Ast_ExprCondOp **out_tern, u8_t max_precedence, bool is_strict_precedence);
\end{lstlisting}

Так например \verb|c_parse_expr_op_infix_prec_rb| проверяет принадлежит ли пунктуатор множеству инфиксных операторов,
затем если приоритет оператора ниже минимального приоритета, то функция выходит с ошибкой, 
если приоритет совпадает с минимальным, то проверяется ассоциативность, если оператор группируется слева на право, то функция выходит с ошибкой.
Также существует флаг \verb|STRICT_PRECEDENCE|, отсекающий случай равных приоритетов.

Следующие функции разбирают выражения, имплементацию их можно посмотреть в приложении[\ref{extras:parse_expr}]

\begin{lstlisting}[language=c]
ParsingError
_c_parse_expr(ParserState *state, C_Ast_Expr **out_expr, u8_t max_precedence, C_ExprFlags flags);
ParsingError
c_parse_expr_assign(ParserState *state, C_Ast_Expr **out_expr);
ParsingError
c_parse_expr_cond(ParserState *state, C_Ast_Expr **out_expr);
ParsingError
c_parse_expr(ParserState *state, C_Ast_Expr **out_expr);
\end{lstlisting}


Графическое изображение AST выражения см. далее[\ref{pratt:right-lean-diag}] и [\ref{graphviz:expr}]

\subsubsection{Парсер Пратта}
\label{pass:parsing:pratt}

Парсер Пратта использует комбинацию циклов и рекурсивных вызовов функций.
Есть хорошая статья, в которой приводится реализация данного парсера на языке Rust\cite{matklad_pratt_parsers}.

Парсер Пратта для инфиксных операторов упрощенно работает по следующему принципу: 

% \begin{lstlisting}[language=python, caption={Псевдокод алгоритма парсера Пратта}]
% def parse_expr(state, max_precedence, flags) -> expr, err:
%     left: Expr
    
% \end{lstlisting}

\begin{enumerate}
    \item Начало определения рекурсивной функции \verb|parse_expr|, 
    принимающей параметры состояния разбора \verb|state| и максимального значения порядка следования \verb|max_precedence|

    \begin{itemize}
        \item Пусть \verb|left: Expr| - переменная типа \verb|Expr|

        \item В \verb|left| записывается первый аргумент или выходим с ошибкой, если его нет или он является именем типа

        \item\label{pratt:alg:loop} Пока есть следующий за аргументом бинарный оператор, удовлетворяющий ограничениям порядка следования и ассоциативности:

        \begin{itemize}
            \item Разбираем оператор записываем в переменную \verb|op: Expr|

            \item\label{pratt:alg:rec-call} Разбираем правый операнд выражения рекурсивным вызовом функции \verb|parse_expr|, 
            со значением \verb|max_precedence| равным порядку следования оператора в переменной \verb|op|

            \item Записываем в \verb|left| значение переменной \verb|op|
        \end{itemize}
    \end{itemize}

    \item Конец определения функции \verb|parse_expr|
\end{enumerate}


Рекурсивные вызовы[\ref{pratt:alg:rec-call}] в алгоритме сами по себе разбирают цепочки операторов с убывающем номером порядка следования
(или одинаковым при условии ассоциативности справа на лево). 
Синтаксические деревья полученные после разбора таких операторов, являются наклоненными направо, как показано на диаграмме[\ref{pratt:right-lean-diag}].

Тогда как цикл[\ref{pratt:alg:loop}] в алгоритме сам по себе разбирает цепочки операторов с оставшимися после рекурсивного вызова номерами порядка следования, 
т.е. возрастающими или одинаковыми при условии ассоциативности слева на право. 
Синтаксические деревья полученные после разбора таких операторов, являются наклоненными налево, как показано на диаграмме[\ref{pratt:left-lean-diag}].



% \begin{figure}[h!]
%     \begin{subfigure}{.5\textwidth}
%         \includegraphics[width=.4\linewidth]{right-lean-diag.png}
%         \centering
%         \caption{AST цепочки операторов с убывающим номером порядка следования}
%         \label{pratt:right-lean-diag}
%     \end{subfigure}

%     \begin{subfigure}{.5\textwidth}
%         \includegraphics[width=.4\linewidth]{left-lean-diag.png}
%         \centering
%         \caption{AST цепочки операторов с возрастающим номером порядка следования}
%         \label{pratt:left-lean-diag}
%     \end{subfigure}
% \end{figure}

% \begin{figure}[h!]
%     \centering
%     \begin{minipage}{.5\textwidth}
%         \includegraphics[width=.4\linewidth]{right-lean-diag.png}
%         \centering
%         \caption{AST цепочки операторов с убывающим номером порядка следования}
%         \label{pratt:right-lean-diag}
%     \end{minipage}
%     \begin{minipage}{.5\textwidth}
%         \includegraphics[width=.4\linewidth]{left-lean-diag.png}
%         \centering
%         \caption{AST цепочки операторов с возрастающим номером порядка следования}
%         \label{pratt:left-lean-diag}
%     \end{minipage}
% \end{figure}

\begin{figure}[h!]
    \includegraphics[width=.4\linewidth]{right-lean-diag.png}
    \centering
    \caption{AST цепочки операторов с убывающим номером порядка следования}
    \label{pratt:right-lean-diag}
\end{figure}
\begin{figure}[h!]
    \includegraphics[width=.4\linewidth]{left-lean-diag.png}
    \centering
    \caption{AST цепочки операторов с возрастающим номером порядка следования}
    \label{pratt:left-lean-diag}
\end{figure}

\FloatBarrier





% \subsection{Определения}

% Виды нод типов:
% \beginminteddef{c}
% enum_def(C_Ast_TypeKind, 
%     C_AST_TYPE_KIND_INVALID,
%     C_AST_TYPE_KIND_IDENT,
%     C_AST_TYPE_KIND_POINTER,
%     C_AST_TYPE_KIND_ARRAY,
%     C_AST_TYPE_KIND_FUNCTION,
%     C_AST_TYPE_KIND_STRUCT,
%     C_AST_TYPE_KIND_UNION,
%     C_AST_TYPE_KIND_ENUM
% )
% \end{minted}

% Виды нод определений:
% \beginminteddef{c}
%     C_AST_DECL_KIND_INVALID,
%     C_AST_DECL_KIND_EMPTY, // ;;
%     C_AST_DECL_KIND_TYPE_DECL, // `struct(enum, union) A {};`
%     C_AST_DECL_KIND_VARIABLE, //  `struct A {} a;`, `int (*foo_p)(void), foo(void);`
%     C_AST_DECL_KIND_FN_DEF, // `int foo(void) {}`
%     C_AST_DECL_KIND_TYPEDEF, // typedef int Foo(void), Arr[3];
% \end{minted}


% Следующие функции разбирают определения, имплементацию их можно посмотреть в приложении[\ref{extras:parse_decl}]
% \beginminteddef{c}
% ParsingError
% c_parse_declarator(ParserState *state, C_Ast_Type **decl_ty_head, C_Ast_Type **decl_ty_leaf, C_Ast_Ident **decl_name);
% ParsingError
% c_parse_direct_declarator(ParserState *state, C_Ast_Type **decl_ty_head, C_Ast_Type **decl_ty_leaf, C_Ast_Ident **decl_name);
% ParsingError
% c_parse_record(ParserState *state, C_Ast_TypeKind struct_or_union_kind, C_Ast_TypeRecord **out_rec);
% ParsingError
% c_parse_type_specifier(ParserState *state, C_Ast_Type **out_ty);
% ParsingError
% c_parse_declaration(ParserState *state, C_Ast_Decl **out_decl);
% \end{minted}

% Функции \verb|c_parse_declarator|, \verb|c_parse_direct_declarator| взаимно рекурсивны, часть их задачи является разбор типов указателей массивов и функций, т.е. выражений вида:
% \mint{c}|int (*(*x)[3])(int a)| 

% Пример визуализации AST для такого определения см. далее[\ref{graphviz:decl}]


% \subsection{Утверждения}

% Виды утверждений:
% \beginminteddef{c}
% enum_def(C_Ast_StmtKind, 
%     C_AST_STMT_KIND_INVALID,
%     C_AST_STMT_KIND_EXPR,

%     C_AST_STMT_KIND_IF,
%     C_AST_STMT_KIND_SWITCH,

%     C_AST_STMT_KIND_LABEL,
%     C_AST_STMT_KIND_CASE,
%     C_AST_STMT_KIND_DEFAULT,

%     C_AST_STMT_KIND_FOR,
%     C_AST_STMT_KIND_WHILE,
%     C_AST_STMT_KIND_DO_WHILE,

%     C_AST_STMT_KIND_GOTO,
%     C_AST_STMT_KIND_CONTINUE,
%     C_AST_STMT_KIND_BREAK,
%     C_AST_STMT_KIND_RETURN,

%     C_AST_STMT_KIND_COMPOUND
% )
% \end{minted}

% \verb|EXPR| - любое выражение оконченное символом \verb|';'| \\
% \verb|COMPOUND| - блок(scope) утверждений, вида

% \beginminteddef{c}
% {
%   // ...
% }
% \end{minted}

% Элементом такого блока является утверждение или определение:
% \beginminteddef{c}
% struct_def(C_Ast_BlockItem, {
%     union {
%     struct {
%         C_AST_NODE_BASE
%     };
%         C_Ast_Decl decl;
%         C_Ast_Stmt stmt;
%     };
    
% })
% \end{minted}


% Остальные виды утверждений очевидно определяются по названию.


% Следующие функции разбирают утверждения, имплементацию их можно посмотреть в приложении[\ref{extras:parse_decl}]
% \beginminteddef{c}
% INLINE
% ParsingError
% c_parse_stmt_expr(ParserState *state, C_Ast_StmtExpr **out_stmt_expr);
% ParsingError
% c_parse_block(ParserState *state, C_SymbolTable *scope, darr_T(C_Ast_BlockItem *) *out_items);
% ParsingError
% c_parse_stmt(ParserState *state, C_Ast_Stmt **out_stmt);
% \end{minted}

% Функция \verb|c_parse_block| добавляет новое пространство имен в \verb|ParserState::env| и заполняет его.


% \subsection{Единица трансляции}

% \verb|C_Ast_TranslationUnit| - AST элемент единица трансляции разбирается схоже с составные утверждением (функция \verb|c_parse_block|),
% однако элементами единицы трансляции являются только определения.

% Следующая функция разбирает единицу трансляции и является самой общей функцией, с нее начинается грамматический разбор.
% \beginminteddef{c}
% ParsingError
% c_parse_translation_unit(ParserState *state, C_Ast_TranslationUnit **out_tr_unit);
% \end{minted}
% Ее реализация приведена в приложении[\ref{extras:parse_tr_unit}].

% Все выше перечисленное организованно в виде прохода:
% \beginminteddef{c}
% bool
% c_translation_unit_parse(C_TranslationUnitData *self) {
%     ASSERT_OK(arena_init(&self->ast_arena, darr_len(self->tokens), ctx_global_alloc));

%     ParserState pstate;

%     _translation_unit_parser_init(self, &pstate);

%         if (IS_ERR(c_parse_translation_unit(&pstate, &self->tr_unit))) {
%         parser_error_print(&pstate);
%         _translation_unit_parser_deinit(&pstate);
%         return false;
%     }

%     _translation_unit_parser_deinit(&pstate);
%     return true;
% }
% \end{minted}

% \clearpage
\subsection{@ директивы}

Привожу грамматику директив, встраиваемою в грамматику Си, приведенную в стандартах:
\begin{lstlisting}[language=bash, caption={Грамматика @ директив}]
at-directive:
    @ directive-name
    @ directive-name (at-directive-args)

at-directive-args:
    at-directive-args at-directive-arg     

at-directive-arg:
    identifier-or-literal

identifier-or-literal:
    identifier
    literal

literal:
    constant
\end{lstlisting}

Таким образом после этапа грамматического разбора мы получаем абстрактное синтаксическое дерево, которое может использовать в дальнейшей работе.