\chapter*{ОПРЕДЕЛЕНИЯ}             % Заголовок
% \addcontentsline{toc}{chapter}{Словарь терминов}  % Добавляем его в оглавление

\ifdefined \acrstyle \else
    \newcommand{\acrstyle}[1]{#1}
\fi


\acrstyle{Лексема(Токен)} - элементарная единица языка с точки зрения его грамматики. Может состоять из нескольких символов.
Примеры: ',' - запятая, '...' - троеточие, 'имя' - идентификатор, '"text"' - строковой литерал

\acrstyle{Лексер} - тип языка программирования, выполняющий лексический разбор

\acrstyle{Парсер} - тип языка программирования, выполняющий грамматический разбор


\acrstyle{Дерево} - связный граф, у каждого элемента которого не более одного предка

\acrstyle{Аллокатор} - абстрактный объект, предоставляющий интерфейс выделения/возврата памяти из кучи(heap)

\acrstyle{Стек(Stack)} -  структура данных, представляющий собой список элементов, организованных по принципу LIFO (last in — first out, последним пришёл — первым вышел).

\acrstyle{Метапрограмма} - программа на некотором языке программирования, которая в процессе своей работы анализирует, трансформирует или генерирует код на том же или другом языке программирования.
Просты словами метапрограмма - программа, результатом работы которой является другая программа.

\acrstyle{Метапрограммы} это своего рода функции высшего порядка\cite{wiki-hof}, отображения между пространствами программ.

\acrstyle{Метапрограммирование} - процесс написания кода метапрограммы.