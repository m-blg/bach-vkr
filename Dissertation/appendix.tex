
\chapter{Проекты}
\paragraph{Проект ecc:} \label{extras:ecc} 
\url{https://github.com/m-blg/c_lang}

\paragraph{Библиотека с основными примитивами:} \label{extras:c-core}
\url{https://github.com/m-blg/c_core}


\chapter{Примитивы}
\section*{Аллокаторы}
\lstinputlisting[language={c},caption={glibc аллокатор},label={extras:glibc_alloc}]{listings/intro/c_alloc.c}

\lstinputlisting[language={c},caption={Арена аллокатор},label={extras:arena_alloc}]{listings/intro/arena.c}

\section*{Форматтер}
\lstinputlisting[language={c},caption={Имплементация основных методов StringFormatter},label={extras:fmt_impl}]{listings/intro/fmt.c}


\chapter{Лексический разбор}

\section*{Определения Си}
\lstinputlisting[language={c},caption={defs.h},label={extras:c_defs}]{listings/lexing/def.h}

\section*{Структура токена Си}
\lstinputlisting[language={c},caption={Структура токена Си},label={extras:c_token}]{listings/lexing/token.c}

\section*{Примеры функций лексического разбора}
\lstinputlisting[language={c},caption={Примеры функций лексического разбора},label={extras:lexer_fns}]{listings/lexing/lexer_fns.c}

\section*{Функции отладочного форматирования токенов}
\lstinputlisting[language={c},caption={Функции отладочного форматирования токенов},label={extras:token_dbg_print}]{listings/lexing/token.c}


\section*{Функция препроцессора}
\lstinputlisting[language={c},caption={Реализация препроцессора},label={extras:pp}]{listings/pp/tokenize.c}


\chapter{Грамматический разбор}
\section*{Абстрактное синтаксическое дерево}
\lstinputlisting[language={c},caption={C AST},label={extras:c_ast}]{listings/parsing/ast.c}

\section*{Функции разбора выражений}
\lstinputlisting[language={c},caption={Функции разбора выражений},label={extras:parse_expr}]{listings/parsing/expr.c}

\section*{Функции разбора определений}
\lstinputlisting[language={c},caption={Функции разбора определений},label={extras:parse_decl}]{listings/parsing/decl.c}

\section*{Функции разбора утдверждений}
\lstinputlisting[language={c},caption={Функции разбора утдверждений},label={extras:parse_stmt}]{listings/parsing/stmt.c}

\section*{Функция разбора единицы трансляции}
\lstinputlisting[language={c},caption={Функция разбора единицы трансляции},label={extras:parse_tr_unit}]{listings/parsing/tr_unit.c}


\chapter{Трансляция Ast}
\section*{Пример функций трансляции AST обратно в Си}
\lstinputlisting[language={c},caption={Функции трансляции AST в Си},label={extras:unparse}]{listings/translation/unparse/unparse.c}

\section*{Пример функций трансляции AST в graphviz}
% label always after caption
\lstinputlisting[language={c},caption={Функции трансляции в graphviz},label={extras:compile-graphviz}]{listings/translation/ast_vis/graphviz.c}



% \chapter{Что-то очень важное}
% \label{app:details}



% \section{секция}
% \[
%     \sin(x) \approx x
% \]

% \begin{lstlisting}
% def EvaluateDiplomas():
%     for each student in Masters:
%         student.Mark := 5
%     for each student in Engineers:
%         if Good(student):
%             student.Mark := 5
%         else:
%             student.Mark := 4
% \end{lstlisting}
% ,caption={Алгоритм оценки дипломных работ}
% \section{другая секция}

% \begin{lstlisting}[language=C, caption={Листинг из внешнего файла}]
%     int x = 4;
% \end{lstlisting}
% % \begin{SmallListing}
% %     int x = 3;
% % \end{SmallListing}

% \begin{ListingEnv}[t]
%     % далее метка для ссылки:
%     % окружение учитывает пробелы и табляции и приеняет их в сответсвии с настройкми
%     % \begin{lstlisting}[language={[ISO]C++}]
%     \begin{lstlisting}[language=C]
% 	#include <iostream>
% 	using namespace std;

% 	int main() //кириллица в комментариях при xelatex и lualatex имеет проблемы с пробелами
% 	{
% 		cout << "Hello, world" << endl; //latin letters in commentaries
% 		system("pause");
% 		return 0;
% 	}
%     \end{lstlisting}
%     \caption{Программа “Hello, world” на \protect\cpp}
%     \label{list:hwbeauty}
% \end{ListingEnv}%

% \chapter{Далее}

% \lstinputlisting[language={C},caption={Листинг из внешнего файла},label={list:external1}]{listings/intro/arena.c}

% \chapter{Основные публикации автора по теме диссертации}
% \label{app:publications}

% % first publation
% \includepdf[
%     pages={-},  % include all pages
%     pagecommand={},  % to include global numbering
%     scale=0.85,  % to leave space for the global page numbers
%     frame,  % adds a frame, optional
% ]{biblio/MyPublications/2019_PRL_acoustic.pdf}

