% Если оч надо это автоматизировать, то смотри здесь
% https://www.overleaf.com/learn/latex/Nomenclatures
%\printnomenclature[3.5cm] % Значение ширины столбца с обозначениями стоит подбирать вручную

\chapter*{ОБОЗНАЧЕНИЯ И СОКРАЩЕНИЯ}             % Заголовок
% \addcontentsline{toc}{chapter}{ОБОЗНАЧЕНИЯ И СОКРАЩЕНИЯ}  % Добавляем его в оглавление

% \newcommand{\acrstyle}[1]{\textbf{#1}}
% \newcommand{\acrstyle}[1]{#1}

\ifdefined \acrstyle \else
    \newcommand{\acrstyle}[1]{#1}
\fi

\acrstyle{CFG} - Context-Free Grammar (контекстно-свободная грамматика)

\acrstyle{AST(АСД)} - Abstract Syntax Tree (абстрактное синтаксическое дерево)


\acrstyle{CLI} -  command line interface, интерфейс командной строки(терминала)

% Заранее ввожу следующие важные определения: