% Если оч надо это автоматизировать, то смотри здесь
% https://www.overleaf.com/learn/latex/Nomenclatures
%\printnomenclature[3.5cm] % Значение ширины столбца с обозначениями стоит подбирать вручную

\chapter*{ОБОЗНАЧЕНИЯ И СОКРАЩЕНИЯ}             % Заголовок
% \addcontentsline{toc}{chapter}{ОБОЗНАЧЕНИЯ И СОКРАЩЕНИЯ}  % Добавляем его в оглавление

% \newcommand{\acrstyle}[1]{\textbf{#1}}
\newcommand{\acrstyle}[1]{#1}


\acrstyle{Лексема(Токен)} - элементарная единица языка с точки зрения его грамматики. Может состоять из нескольких символов.
Примеры: ',' - запятая, '...' - троеточие, 'имя' - идентификатор, '"text"' - строковой литерал

\acrstyle{CFG} - Context-Free Grammar (контекстно-свободная грамматика)

\acrstyle{AST(АСД)} - Abstract Syntax Tree (абстрактное синтаксическое дерево)

\acrstyle{Лексер} - тип языка программирования, выполняющий лексический разбор

\acrstyle{Парсер} - тип языка программирования, выполняющий грамматический разбор


\acrstyle{Дерево} - связный граф, у каждого элемента которого не более одного предка

\acrstyle{Аллокатор} - абстрактный объект, предоставляющий интерфейс выделения/возврата памяти из кучи(heap)

\acrstyle{Стек(Stack)} -  структура данных, представляющий собой список элементов, организованных по принципу LIFO (last in — first out, последним пришёл — первым вышел).

\acrstyle{CLI} -  command line interface, интерфейс командной строки(терминала)